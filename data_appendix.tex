% Options for packages loaded elsewhere
\PassOptionsToPackage{unicode}{hyperref}
\PassOptionsToPackage{hyphens}{url}
%
\documentclass[
]{article}
\usepackage{lmodern}
\usepackage{amssymb,amsmath}
\usepackage{ifxetex,ifluatex}
\ifnum 0\ifxetex 1\fi\ifluatex 1\fi=0 % if pdftex
  \usepackage[T1]{fontenc}
  \usepackage[utf8]{inputenc}
  \usepackage{textcomp} % provide euro and other symbols
\else % if luatex or xetex
  \usepackage{unicode-math}
  \defaultfontfeatures{Scale=MatchLowercase}
  \defaultfontfeatures[\rmfamily]{Ligatures=TeX,Scale=1}
\fi
% Use upquote if available, for straight quotes in verbatim environments
\IfFileExists{upquote.sty}{\usepackage{upquote}}{}
\IfFileExists{microtype.sty}{% use microtype if available
  \usepackage[]{microtype}
  \UseMicrotypeSet[protrusion]{basicmath} % disable protrusion for tt fonts
}{}
\makeatletter
\@ifundefined{KOMAClassName}{% if non-KOMA class
  \IfFileExists{parskip.sty}{%
    \usepackage{parskip}
  }{% else
    \setlength{\parindent}{0pt}
    \setlength{\parskip}{6pt plus 2pt minus 1pt}}
}{% if KOMA class
  \KOMAoptions{parskip=half}}
\makeatother
\usepackage{xcolor}
\IfFileExists{xurl.sty}{\usepackage{xurl}}{} % add URL line breaks if available
\IfFileExists{bookmark.sty}{\usepackage{bookmark}}{\usepackage{hyperref}}
\hypersetup{
  pdftitle={Data Appendix to The Influence of Individual Characterisitcs on Public Transportation Planning},
  pdfauthor={Iris Zhong},
  hidelinks,
  pdfcreator={LaTeX via pandoc}}
\urlstyle{same} % disable monospaced font for URLs
\usepackage[margin=1in]{geometry}
\usepackage{color}
\usepackage{fancyvrb}
\newcommand{\VerbBar}{|}
\newcommand{\VERB}{\Verb[commandchars=\\\{\}]}
\DefineVerbatimEnvironment{Highlighting}{Verbatim}{commandchars=\\\{\}}
% Add ',fontsize=\small' for more characters per line
\usepackage{framed}
\definecolor{shadecolor}{RGB}{248,248,248}
\newenvironment{Shaded}{\begin{snugshade}}{\end{snugshade}}
\newcommand{\AlertTok}[1]{\textcolor[rgb]{0.94,0.16,0.16}{#1}}
\newcommand{\AnnotationTok}[1]{\textcolor[rgb]{0.56,0.35,0.01}{\textbf{\textit{#1}}}}
\newcommand{\AttributeTok}[1]{\textcolor[rgb]{0.77,0.63,0.00}{#1}}
\newcommand{\BaseNTok}[1]{\textcolor[rgb]{0.00,0.00,0.81}{#1}}
\newcommand{\BuiltInTok}[1]{#1}
\newcommand{\CharTok}[1]{\textcolor[rgb]{0.31,0.60,0.02}{#1}}
\newcommand{\CommentTok}[1]{\textcolor[rgb]{0.56,0.35,0.01}{\textit{#1}}}
\newcommand{\CommentVarTok}[1]{\textcolor[rgb]{0.56,0.35,0.01}{\textbf{\textit{#1}}}}
\newcommand{\ConstantTok}[1]{\textcolor[rgb]{0.00,0.00,0.00}{#1}}
\newcommand{\ControlFlowTok}[1]{\textcolor[rgb]{0.13,0.29,0.53}{\textbf{#1}}}
\newcommand{\DataTypeTok}[1]{\textcolor[rgb]{0.13,0.29,0.53}{#1}}
\newcommand{\DecValTok}[1]{\textcolor[rgb]{0.00,0.00,0.81}{#1}}
\newcommand{\DocumentationTok}[1]{\textcolor[rgb]{0.56,0.35,0.01}{\textbf{\textit{#1}}}}
\newcommand{\ErrorTok}[1]{\textcolor[rgb]{0.64,0.00,0.00}{\textbf{#1}}}
\newcommand{\ExtensionTok}[1]{#1}
\newcommand{\FloatTok}[1]{\textcolor[rgb]{0.00,0.00,0.81}{#1}}
\newcommand{\FunctionTok}[1]{\textcolor[rgb]{0.00,0.00,0.00}{#1}}
\newcommand{\ImportTok}[1]{#1}
\newcommand{\InformationTok}[1]{\textcolor[rgb]{0.56,0.35,0.01}{\textbf{\textit{#1}}}}
\newcommand{\KeywordTok}[1]{\textcolor[rgb]{0.13,0.29,0.53}{\textbf{#1}}}
\newcommand{\NormalTok}[1]{#1}
\newcommand{\OperatorTok}[1]{\textcolor[rgb]{0.81,0.36,0.00}{\textbf{#1}}}
\newcommand{\OtherTok}[1]{\textcolor[rgb]{0.56,0.35,0.01}{#1}}
\newcommand{\PreprocessorTok}[1]{\textcolor[rgb]{0.56,0.35,0.01}{\textit{#1}}}
\newcommand{\RegionMarkerTok}[1]{#1}
\newcommand{\SpecialCharTok}[1]{\textcolor[rgb]{0.00,0.00,0.00}{#1}}
\newcommand{\SpecialStringTok}[1]{\textcolor[rgb]{0.31,0.60,0.02}{#1}}
\newcommand{\StringTok}[1]{\textcolor[rgb]{0.31,0.60,0.02}{#1}}
\newcommand{\VariableTok}[1]{\textcolor[rgb]{0.00,0.00,0.00}{#1}}
\newcommand{\VerbatimStringTok}[1]{\textcolor[rgb]{0.31,0.60,0.02}{#1}}
\newcommand{\WarningTok}[1]{\textcolor[rgb]{0.56,0.35,0.01}{\textbf{\textit{#1}}}}
\usepackage{graphicx,grffile}
\makeatletter
\def\maxwidth{\ifdim\Gin@nat@width>\linewidth\linewidth\else\Gin@nat@width\fi}
\def\maxheight{\ifdim\Gin@nat@height>\textheight\textheight\else\Gin@nat@height\fi}
\makeatother
% Scale images if necessary, so that they will not overflow the page
% margins by default, and it is still possible to overwrite the defaults
% using explicit options in \includegraphics[width, height, ...]{}
\setkeys{Gin}{width=\maxwidth,height=\maxheight,keepaspectratio}
% Set default figure placement to htbp
\makeatletter
\def\fps@figure{htbp}
\makeatother
\setlength{\emergencystretch}{3em} % prevent overfull lines
\providecommand{\tightlist}{%
  \setlength{\itemsep}{0pt}\setlength{\parskip}{0pt}}
\setcounter{secnumdepth}{5}

\title{Data Appendix to The Influence of Individual Characterisitcs on Public
Transportation Planning}
\author{Iris Zhong}
\date{}

\begin{document}
\maketitle

{
\setcounter{tocdepth}{2}
\tableofcontents
}
\hypertarget{raw-data}{%
\section{Raw data}\label{raw-data}}

\hypertarget{gwinnett-county-2019-referendum-dataset}{%
\subsection{Gwinnett County 2019 Referendum
Dataset}\label{gwinnett-county-2019-referendum-dataset}}

\textbf{Citation:} Results---Gwinnett---Election Night Reporting.
(n.d.). Retrieved March 11, 2020, from
\url{https://results.enr.clarityelections.com/GA/Gwinnett/94961/Web02.225391/\#/}\\
\textbf{DOI:} N/A\\
\textbf{Date Downloaded:} Mar 11, 2020\\
\textbf{Filename:} raw\_data/vote\_result.xls\\
\textbf{Unit of observation:} Precinct\\
\textbf{Dates covered:} Mar 19, 2019

\hypertarget{to-obtain-a-copy}{%
\subsubsection{To obtain a copy}\label{to-obtain-a-copy}}

Users can visit the website that displays election results at Gwinnett
County at
\url{https://results.enr.clarityelections.com/GA/Gwinnett/94961/Web02.225391/\#/}
and choose the \textbf{Detail XLS} link at the bottom right corner.

The xls file contains three spreadsheets. I will be using the second and
the third sheets.

\hypertarget{importable-version}{%
\subsubsection{Importable version}\label{importable-version}}

\textbf{Filename:} importable\_data/vote\_result\_importable.xlsx

The raw dataset is hard to be imported to R directly because of the
following reasons. First, it has three sheets. Second, the top two rows
do not contain any useful information or should be incorporated into the
next row. The file uses the extension .xls, which is incompatible with
R. Therefore, an importable version of the dataset was created.

Here are the steps:

\begin{enumerate}
\def\labelenumi{\arabic{enumi}.}
\tightlist
\item
  Open the original files in Excel.
\item
  Move the turnout rate for each precinct displayed in the second
  spreadsheet to the third spreadsheet.
\item
  Delete the first and the second spreadsheets.
\item
  Remove the top two rows of the third spreadsheet.
\item
  Rename the columns to reflect whether the vote was for or against the
  proposal.
\item
  Change the extension from .xls to .xlsx.
\end{enumerate}

\hypertarget{variable-descriptions}{%
\subsubsection{Variable descriptions}\label{variable-descriptions}}

I cannot find any information that describes the variables in this
dataset. Therefore, the description below is my understanding.

\begin{itemize}
\tightlist
\item
  \textbf{precinct:} The name of the precinct.
\item
  \textbf{registered\_voters:} The number of registered voters in the
  precinct.
\item
  \textbf{total\_votes:} The number of votes received in this
  referendum.
\item
  \textbf{voter\_turnout:} The percentage of voters who voted in this
  referendum. (\(total\_votes/reigstered\_voters\))
\item
  \textbf{election\_day\_yes:} The number of people who voted yes during
  the election day.
\item
  \textbf{absentee\_mail\_yes:} The number of people who voted yes by
  mailing paper ballots prior to the election day.
\item
  \textbf{advance\_in\_person\_1\_yes:} The number of people who voted
  yes prior to the election day.
\item
  \textbf{advance\_in\_person\_2\_yes:} The number of people who voted
  yes prior to the election day. The difference between this variable
  from the previous one is not clear. My speculation is they record
  people voting on different days before election.
\item
  \textbf{provisional\_yes:} The number of people who voted yes but had
  questions in their eligibility.
\item
  \textbf{votes\_yes:} The number of people who voted yes in total. (the
  sum of the previous five variables)
\item
  \textbf{election\_day\_no:} The number of people who voted no during
  the election day.
\item
  \textbf{absentee\_mail\_no:} The number of people who voted no by
  mailing paper ballots prior to the election day.
\item
  \textbf{advance\_in\_person\_1\_no:} The number of people who voted no
  prior to the election day.
\item
  \textbf{advance\_in\_person\_2\_no:} The number of people who voted no
  prior to the election day. The difference between this variable from
  the previous one is not clear. My speculation is they record people
  voting on different days before election.
\item
  \textbf{provisional\_no:} The number of people who voted no but had
  questions in their eligibility.
\item
  \textbf{votes\_no:} The number of people who voted no in total. (the
  sum of the previous five variables)
\end{itemize}

\hypertarget{data-import-code-and-summary}{%
\subsubsection{Data import code and
summary}\label{data-import-code-and-summary}}

\begin{Shaded}
\begin{Highlighting}[]
\NormalTok{vote_result <-}\StringTok{ }\KeywordTok{read_excel}\NormalTok{(}\StringTok{"importable_data/vote_result_importable.xlsx"}\NormalTok{, }
    \DataTypeTok{col_types =} \KeywordTok{c}\NormalTok{(}\StringTok{"text"}\NormalTok{, }\StringTok{"numeric"}\NormalTok{, }\StringTok{"numeric"}\NormalTok{, }
        \StringTok{"numeric"}\NormalTok{, }\StringTok{"numeric"}\NormalTok{, }\StringTok{"numeric"}\NormalTok{, }
        \StringTok{"numeric"}\NormalTok{, }\StringTok{"numeric"}\NormalTok{, }\StringTok{"numeric"}\NormalTok{, }\StringTok{"numeric"}\NormalTok{, }
        \StringTok{"numeric"}\NormalTok{, }\StringTok{"numeric"}\NormalTok{, }\StringTok{"numeric"}\NormalTok{, }\StringTok{"numeric"}\NormalTok{, }\StringTok{"numeric"}\NormalTok{, }
        \StringTok{"numeric"}\NormalTok{))}
\KeywordTok{View}\NormalTok{(vote_result)}
\KeywordTok{export_summary_table}\NormalTok{(}\KeywordTok{dfSummary}\NormalTok{(vote_result))}
\end{Highlighting}
\end{Shaded}

\hypertarget{gwinnett-county-census-data}{%
\subsection{Gwinnett County Census
Data}\label{gwinnett-county-census-data}}

\textbf{Citation:} U.S. Census Bureau. (2018). American Community
Survey.
\url{https://data.census.gov/cedsci/table?d=ACS\%205-Year\%20Estimates\%20Data\%20Profiles\&table=DP05\&tid=ACSDP5Y2018.DP05\&g=0500000US13135,13135.140000\&hidePreview=true\&t=Age\%20and\%20Sex\%3AHousing\%3AHousing\%20Units\%3ARace\%20and\%20Ethnicity}\\
\textbf{DOI:} N/A\\
\textbf{Date Downloaded:} Mar 19, 2020\\
\textbf{Filename:} N/A\\
\textbf{Unit of observation:} Census tract\\
\textbf{Dates covered:} 2018 (5-year estimate)

\hypertarget{to-obtain-a-copy-1}{%
\subsubsection{To obtain a copy}\label{to-obtain-a-copy-1}}

Users can obtain a copy of the dataset from an R package
\texttt{tidycensus}.

\textbf{Citation:} Walker, K., Eberwein, K., \& Herman, M. (2020).
tidycensus: Load US Census Boundary and Attribute Data as ``tidyverse''
and ``sf''-Ready Data Frames (Version 0.9.6) {[}Computer software{]}.
\url{https://CRAN.R-project.org/package=tidycensus}

Below are the steps to use \texttt{tidycensus} to obtain the data:

\begin{enumerate}
\def\labelenumi{\arabic{enumi}.}
\tightlist
\item
  Install the package \texttt{tidycensus} in R.
\item
  If haven't, register to get an API key in order to download data from
  the package. The key can be acquired at
  \url{http://api.census.gov/data/key_signup.html}.
\item
  Load the key into R with the following code:\\
  \texttt{census\_api\_key(key,\ install\ =\ TRUE)}~\\
\item
  Load the library in R. Execute the code chunk below to get the data
  frame \texttt{acs18}. \texttt{acs18} shows all of the variables
  present in ACS-5 2018 survey and their IDs.
\end{enumerate}

\begin{Shaded}
\begin{Highlighting}[]
\KeywordTok{library}\NormalTok{(tidycensus)}
\NormalTok{acs18 <-}\StringTok{ }\KeywordTok{load_variables}\NormalTok{(}\DecValTok{2018}\NormalTok{, }\StringTok{"acs5"}\NormalTok{, }\DataTypeTok{cache =} \OtherTok{TRUE}\NormalTok{)}
\end{Highlighting}
\end{Shaded}

\begin{enumerate}
\def\labelenumi{\arabic{enumi}.}
\setcounter{enumi}{4}
\tightlist
\item
  Search for desirable variables in acs18 and record their IDs. The
  selected variables are: population, median income, median age, white
  population, the number of people who work, the number of people who
  commute by car, the number of people who commute by public
  transportation, the number of people who commute by subway, the number
  of people who go to work by bike, the number of people who walk to
  work, the number of people who use other transportation means, and the
  number of people who work at home. Subway is a subcategory of public
  transportation, but since it is particularly important in this
  project, it is also selected. Besides, variables that reflect people's
  travel time to work are potentially useful.
\end{enumerate}

\begin{Shaded}
\begin{Highlighting}[]
\NormalTok{cbdata <-}\StringTok{ }\KeywordTok{get_acs}\NormalTok{(}\DataTypeTok{geography =} \StringTok{"tract"}\NormalTok{,}
                  \DataTypeTok{variables =} \KeywordTok{c}\NormalTok{(}\DataTypeTok{total =} \StringTok{"B01001_001"}\NormalTok{,}
                                \DataTypeTok{medincome =} \StringTok{"B19013_001"}\NormalTok{,}
                                \DataTypeTok{medage =} \StringTok{"B01002_001"}\NormalTok{,}
                                \DataTypeTok{white =} \StringTok{"B01001A_001"}\NormalTok{,}
                                \DataTypeTok{transportation_total =} \StringTok{"B08006_001"}\NormalTok{,}
                                \DataTypeTok{car =} \StringTok{"B08006_002"}\NormalTok{,}
                                \DataTypeTok{public =} \StringTok{"B08006_008"}\NormalTok{,}
                                \DataTypeTok{subway =} \StringTok{"B08006_011"}\NormalTok{,}
                                \DataTypeTok{bike =} \StringTok{"B08006_014"}\NormalTok{,}
                                \DataTypeTok{walk =} \StringTok{"B08006_015"}\NormalTok{,}
                                \DataTypeTok{other_transport =} \StringTok{"B08006_016"}\NormalTok{,}
                                \DataTypeTok{no_transport =} \StringTok{"B08006_017"}\NormalTok{,}
                                \DataTypeTok{time_total =} \StringTok{"B08012_001"}\NormalTok{,}
                                \DataTypeTok{time_less_5 =} \StringTok{"B08012_002"}\NormalTok{,}
                                \DataTypeTok{time_5_9 =} \StringTok{"B08012_003"}\NormalTok{,}
                                \DataTypeTok{time_10_14 =} \StringTok{"B08012_004"}\NormalTok{,}
                                \DataTypeTok{time_15_19 =} \StringTok{"B08012_005"}\NormalTok{,}
                                \DataTypeTok{time_20_24 =} \StringTok{"B08012_006"}\NormalTok{,}
                                \DataTypeTok{time_25_29 =} \StringTok{"B08012_007"}\NormalTok{,}
                                \DataTypeTok{time_30_34 =} \StringTok{"B08012_008"}\NormalTok{,}
                                \DataTypeTok{time_35_39 =} \StringTok{"B08012_009"}\NormalTok{,}
                                \DataTypeTok{time_40_44 =} \StringTok{"B08012_010"}\NormalTok{,}
                                \DataTypeTok{time_45_59 =} \StringTok{"B08012_011"}\NormalTok{,}
                                \DataTypeTok{time_60_89 =} \StringTok{"B08012_012"}\NormalTok{,}
                                \DataTypeTok{time_more_90 =} \StringTok{"B08012_013"}\NormalTok{),}
                  \DataTypeTok{state =} \StringTok{"GA"}\NormalTok{,}
                  \DataTypeTok{county =} \StringTok{"Gwinnett"}\NormalTok{,}
                  \DataTypeTok{year =} \DecValTok{2018}\NormalTok{)}
\end{Highlighting}
\end{Shaded}

\begin{verbatim}
## Getting data from the 2014-2018 5-year ACS
\end{verbatim}

The code above constructs a data frame called cbdata by calling the
function \texttt{get\_acs()}, which pulls the data from the American
Community Survey. Inside the function, the unit of measurement is
specified by the \texttt{geography} argument. In this case, select
\textbf{tract} for census tract. Put all the chosen variables in the
\texttt{variables} argument. Finally, address \texttt{state}
(\textbf{GA}), \texttt{county} (\textbf{Gwinnett}), and \texttt{year} of
survey (\textbf{2018}).

\hypertarget{data-wrangling}{%
\subsubsection{Data wrangling}\label{data-wrangling}}

The current data frame requires modifications. First, each row displays
one variable from one tract. However, to make tract as the unit of
measurement, each row should include all the variables of one tract.
Second, instead of the actual number of people who are white, the
percentage of white population is more informative. Similarly, the
percentage of people who go to work by certain transportation should
also be calculated. Finally, travel time data is mostly grouped by a
5-minute band, which is too detailed for this project. It will be
categorized with a wider range to reduce variables. After consolidation,
the percentages will be calculated.

Here are the steps of data wrangling:

\begin{enumerate}
\def\labelenumi{\arabic{enumi}.}
\tightlist
\item
  Remove the margin of error for each measurement (\textbf{moe}),
  because it is not useful in later analyses.
\end{enumerate}

\begin{Shaded}
\begin{Highlighting}[]
\NormalTok{cbdata_moe <-}\StringTok{ }\NormalTok{cbdata }\OperatorTok
\StringTok{  }\KeywordTok{select}\NormalTok{ (}\OperatorTok{-}\NormalTok{moe)}
\end{Highlighting}
\end{Shaded}

\begin{enumerate}
\def\labelenumi{\arabic{enumi}.}
\setcounter{enumi}{1}
\tightlist
\item
  Use \texttt{pivot\_wider()} to transpose the data.
\end{enumerate}

\begin{Shaded}
\begin{Highlighting}[]
\NormalTok{cbdata_wider <-}\StringTok{ }\NormalTok{cbdata_moe }\OperatorTok
\StringTok{  }\KeywordTok{pivot_wider}\NormalTok{(}\DataTypeTok{names_from =}\NormalTok{ variable,}
              \DataTypeTok{values_from =}\NormalTok{ estimate)}
\end{Highlighting}
\end{Shaded}

\begin{enumerate}
\def\labelenumi{\arabic{enumi}.}
\setcounter{enumi}{2}
\tightlist
\item
  Calculate the percentage of white population (\(white/total\)).
\end{enumerate}

\begin{Shaded}
\begin{Highlighting}[]
\NormalTok{cbdata_white <-}\StringTok{ }\NormalTok{cbdata_wider }\OperatorTok
\StringTok{  }\KeywordTok{mutate}\NormalTok{(}\DataTypeTok{white_pct =}\NormalTok{ white }\OperatorTok{/}\StringTok{ }\NormalTok{total) }\OperatorTok
\StringTok{  }\KeywordTok{select}\NormalTok{ (}\OperatorTok{-}\NormalTok{white)}
\end{Highlighting}
\end{Shaded}

\begin{enumerate}
\def\labelenumi{\arabic{enumi}.}
\setcounter{enumi}{3}
\tightlist
\item
  Calculate the percentage of people using each transportation method.
\end{enumerate}

\begin{Shaded}
\begin{Highlighting}[]
\NormalTok{cbdata_transport <-}\StringTok{ }\NormalTok{cbdata_white }\OperatorTok
\StringTok{  }\KeywordTok{mutate}\NormalTok{(}\DataTypeTok{car_pct =}\NormalTok{ car }\OperatorTok{/}\StringTok{ }\NormalTok{transportation_total,}
         \DataTypeTok{public_pct =}\NormalTok{ public }\OperatorTok{/}\StringTok{ }\NormalTok{transportation_total,}
         \DataTypeTok{subway_pct =}\NormalTok{ subway }\OperatorTok{/}\StringTok{ }\NormalTok{transportation_total,}
         \DataTypeTok{bike_pct =}\NormalTok{ bike }\OperatorTok{/}\StringTok{ }\NormalTok{transportation_total,}
         \DataTypeTok{walk_pct =}\NormalTok{ walk }\OperatorTok{/}\StringTok{ }\NormalTok{transportation_total,}
         \DataTypeTok{other_pct =}\NormalTok{ other_transport }\OperatorTok{/}\StringTok{ }\NormalTok{transportation_total,}
         \DataTypeTok{no_pct =}\NormalTok{ no_transport }\OperatorTok{/}\StringTok{ }\NormalTok{transportation_total) }\OperatorTok
\StringTok{  }\KeywordTok{select}\NormalTok{(}\OperatorTok{-}\KeywordTok{c}\NormalTok{(car, public, subway, bike, walk, other_transport, no_transport))}
\end{Highlighting}
\end{Shaded}

\begin{enumerate}
\def\labelenumi{\arabic{enumi}.}
\setcounter{enumi}{4}
\tightlist
\item
  Combine the ranges of travel time data and calculate the percentages.
\end{enumerate}

\begin{Shaded}
\begin{Highlighting}[]
\NormalTok{cbdata_tidy <-}\StringTok{ }\NormalTok{cbdata_transport }\OperatorTok
\StringTok{  }\KeywordTok{mutate}\NormalTok{(}\DataTypeTok{time_0_29_pct =}\NormalTok{ (time_less_}\DecValTok{5} \OperatorTok{+}\StringTok{ }\NormalTok{time_}\DecValTok{10}\NormalTok{_}\DecValTok{14} \OperatorTok{+}\StringTok{ }\NormalTok{time_}\DecValTok{15}\NormalTok{_}\DecValTok{19} \OperatorTok{+}\StringTok{ }
\StringTok{                            }\NormalTok{time_}\DecValTok{20}\NormalTok{_}\DecValTok{24} \OperatorTok{+}\StringTok{ }\NormalTok{time_}\DecValTok{25}\NormalTok{_}\DecValTok{29}\NormalTok{) }\OperatorTok{/}\StringTok{ }\NormalTok{time_total,}
         \DataTypeTok{time_30_59_pct =}\NormalTok{ (time_}\DecValTok{30}\NormalTok{_}\DecValTok{34} \OperatorTok{+}\StringTok{ }\NormalTok{time_}\DecValTok{35}\NormalTok{_}\DecValTok{39} \OperatorTok{+}\StringTok{ }\NormalTok{time_}\DecValTok{40}\NormalTok{_}\DecValTok{44} \OperatorTok{+}\StringTok{ }
\StringTok{                             }\NormalTok{time_}\DecValTok{45}\NormalTok{_}\DecValTok{59}\NormalTok{) }\OperatorTok{/}\StringTok{ }\NormalTok{time_total,}
         \DataTypeTok{time_60_89_pct =}\NormalTok{ time_}\DecValTok{60}\NormalTok{_}\DecValTok{89} \OperatorTok{/}\StringTok{ }\NormalTok{time_total,}
         \DataTypeTok{time_more_90_pct =}\NormalTok{ time_more_}\DecValTok{90} \OperatorTok{/}\StringTok{ }\NormalTok{time_total) }\OperatorTok
\StringTok{  }\KeywordTok{select}\NormalTok{(}\OperatorTok{-}\KeywordTok{c}\NormalTok{(time_less_}\DecValTok{5}\NormalTok{, time_}\DecValTok{5}\NormalTok{_}\DecValTok{9}\NormalTok{, time_}\DecValTok{10}\NormalTok{_}\DecValTok{14}\NormalTok{, time_}\DecValTok{15}\NormalTok{_}\DecValTok{19}\NormalTok{, time_}\DecValTok{20}\NormalTok{_}\DecValTok{24}\NormalTok{,}
\NormalTok{            time_}\DecValTok{25}\NormalTok{_}\DecValTok{29}\NormalTok{, time_}\DecValTok{30}\NormalTok{_}\DecValTok{34}\NormalTok{, time_}\DecValTok{35}\NormalTok{_}\DecValTok{39}\NormalTok{, time_}\DecValTok{40}\NormalTok{_}\DecValTok{44}\NormalTok{, time_}\DecValTok{45}\NormalTok{_}\DecValTok{59}\NormalTok{,}
\NormalTok{            time_}\DecValTok{60}\NormalTok{_}\DecValTok{89}\NormalTok{, time_more_}\DecValTok{90}\NormalTok{))}
\end{Highlighting}
\end{Shaded}

\hypertarget{variable-descriptions-1}{%
\subsubsection{Variable descriptions}\label{variable-descriptions-1}}

\begin{itemize}
\tightlist
\item
  \textbf{GEOID:} The geographic identifier of the census tract.
\item
  \textbf{NAME:} The name of the census tract.
\item
  \textbf{total:} The total population of the tract.
\item
  \textbf{medage:} The median age of the population in the tract.
\item
  \textbf{medincome:} The median income of the population in the tract.
\item
  \textbf{white\_pct:} The percentage of white population in the tract.
\item
  \textbf{transportation\_total:} The number of people who were sampled
  in the transportation survey.
\item
  \textbf{car\_pct:} The percentage of people who go to work by car,
  truck or van.
\item
  \textbf{public\_pct:} The percentage of people who go to work by
  public transportation (excluding taxi or cab).
\item
  \textbf{subway\_pct:} The percentage of people who go to work by
  subway or elevated.
\item
  \textbf{bike\_pct:} The percentage of people who go to work by bike.
\item
  \textbf{walk\_pct:} The percentage of people who go to work on foot.
\item
  \textbf{other\_pct:} The percentage of people who go to work by other
  transportation means such as taxi, cab and motorcycle.
\item
  \textbf{no\_pct:} The percentage of people who work at home (i.e.~no
  transportation needed).
\item
  \textbf{time\_total:} The number of people who were sampled in the
  travel time to work survey.
\item
  \textbf{time\_0\_29\_pct:} The percentage of people who travel less
  than 30 minutes to work.
\item
  \textbf{time\_30\_59\_pct:} The percentage of people who travel
  between 30 and 59 minutes to work.
\item
  \textbf{time\_60\_89\_pct:} The percentage of people who travel
  between 60 and 89 minutes to work.
\item
  \textbf{time\_more\_90\_pct:} The percentage of people who travel more
  than 90 minutes to work.
\end{itemize}

\hypertarget{data-summary}{%
\subsubsection{Data summary}\label{data-summary}}

\begin{Shaded}
\begin{Highlighting}[]
\KeywordTok{View}\NormalTok{(cbdata_tidy)}
\KeywordTok{export_summary_table}\NormalTok{(}\KeywordTok{dfSummary}\NormalTok{(cbdata_tidy))}
\end{Highlighting}
\end{Shaded}

\hypertarget{presidential-election-data-at-gwinnett-county}{%
\subsection{2016 Presidential Election Data at Gwinnett
County}\label{presidential-election-data-at-gwinnett-county}}

\textbf{Citation:} MIT Election Data and Science Lab, 2018, U.S.
President Precinct-Level Returns 2016, \emph{Harvard Dataverse, V11,}
UNF:6:hQyVqHW+vTFnAW2jYIOy/Q== {[}fileUNF{]}\\
\textbf{DOI:} \url{doi:10.7910/DVN/LYWX3D}\\
\textbf{Date Downloaded:} Mar 19, 2020\\
\textbf{Filename(s):} N/A\\
\textbf{Unit of observation:} Precinct\\
\textbf{Dates covered:} November 8, 2016

\hypertarget{to-obtain-a-copy-2}{%
\subsubsection{To obtain a copy}\label{to-obtain-a-copy-2}}

Users can obtain a copy of the dataset at
\url{https://dataverse.harvard.edu/dataset.xhtml?persistentId=doi:10.7910/DVN/LYWX3D}.
Under the \textbf{Files} tab, select and click the download button of
the first file \textbf{2016-precinct-president.tab}. The description of
the variables can be found in the second file,
\textbf{codebook-2016-precinct-president.md}.

However, the data file is not included in the raw\_data folder, because
it is too large to be uploaded to github.

\hypertarget{importable-version-1}{%
\subsubsection{Importable version}\label{importable-version-1}}

\textbf{Filename:} importable\_data/election\_result\_importable.csv

The original file contains data from all the precincts in the United
States. Therefore, it is too large to be loaded to R or github.
Therefore, filtering is necessary before importing.

Here are the steps of filtering:

\begin{enumerate}
\def\labelenumi{\arabic{enumi}.}
\tightlist
\item
  Open the file in excel.
\item
  Click on \textbf{Sort \& Filter} button; click \textbf{Filter}.
\item
  Go to column H -- \textbf{county\_name} and click on the small
  triangle in cell H1.
\item
  Find and select \textbf{Gwinnett County} in the drop-down menu.
\item
  Copy and paste this subset of data into another file and put it into
  the importable data folder.
\end{enumerate}

\hypertarget{data-import-and-wrangling}{%
\subsubsection{Data import and
wrangling}\label{data-import-and-wrangling}}

\begin{Shaded}
\begin{Highlighting}[]
\NormalTok{election <-}\StringTok{ }\KeywordTok{read_csv}\NormalTok{(}\StringTok{"importable_data/election_result_importable.csv"}\NormalTok{)}
\end{Highlighting}
\end{Shaded}

The data needs more cleaning in R because firstly, it has unnecessary
variables. Secondly, each row in this dataset represents the number of
people voting for one particular candidate by one particular mode in a
precinct. The ideal dataset will have the voting results for each
candidate at one precinct in one row to make precinct as the unit of
measurement.

Here are the steps of data wrangling:

\begin{enumerate}
\def\labelenumi{\arabic{enumi}.}
\tightlist
\item
  Select the useful variables: \textbf{precinct}, \textbf{candidate},
  \textbf{votes}, \textbf{mode}. For more information about the removed
  variables, check out the codebook in the raw data folder.
\end{enumerate}

\begin{Shaded}
\begin{Highlighting}[]
\NormalTok{election_variables <-}\StringTok{ }\NormalTok{election }\OperatorTok
\StringTok{  }\KeywordTok{select}\NormalTok{(precinct, candidate, votes, mode)}
\end{Highlighting}
\end{Shaded}

\begin{enumerate}
\def\labelenumi{\arabic{enumi}.}
\setcounter{enumi}{1}
\tightlist
\item
  Summarize the number of votes for each candidate in a precinct. This
  is done by adding across different modes of votes (election day,
  absentee by mail, advance in person, and provisional). Then calculate
  the percentage of votes for each candidate.
\end{enumerate}

\begin{Shaded}
\begin{Highlighting}[]
\NormalTok{election_mode <-}\StringTok{ }\NormalTok{election_variables }\OperatorTok
\StringTok{  }\KeywordTok{group_by}\NormalTok{(precinct, candidate) }\OperatorTok
\StringTok{  }\KeywordTok{summarize}\NormalTok{(}\DataTypeTok{votes =} \KeywordTok{sum}\NormalTok{(votes))}
\end{Highlighting}
\end{Shaded}

\begin{enumerate}
\def\labelenumi{\arabic{enumi}.}
\setcounter{enumi}{2}
\tightlist
\item
  Remove the write-in votes because they don't belong to any specific
  precincts; transpose the data; finally, calculate the precentage of
  votes for each candidate.
\end{enumerate}

\begin{Shaded}
\begin{Highlighting}[]
\NormalTok{election_tidy <-}\StringTok{ }\KeywordTok{as.data.frame}\NormalTok{(election_mode }\OperatorTok
\StringTok{  }\KeywordTok{filter}\NormalTok{(precinct }\OperatorTok{!=}\StringTok{ "Write-ins"}\NormalTok{) }\OperatorTok
\StringTok{  }\KeywordTok{pivot_wider}\NormalTok{(}\DataTypeTok{names_from =}\NormalTok{ candidate,}
              \DataTypeTok{values_from =}\NormalTok{ votes) }\OperatorTok
\StringTok{  }\KeywordTok{mutate}\NormalTok{(}\DataTypeTok{total =} \StringTok{`}\DataTypeTok{Donald Trump}\StringTok{`} \OperatorTok{+}\StringTok{ `}\DataTypeTok{Hillary Clinton}\StringTok{`} \OperatorTok{+}\StringTok{ `}\DataTypeTok{Gary Johnson}\StringTok{`}\NormalTok{) }\OperatorTok
\StringTok{  }\KeywordTok{select}\NormalTok{(precinct, }\DataTypeTok{trump_votes =} \StringTok{`}\DataTypeTok{Donald Trump}\StringTok{`}\NormalTok{, }\DataTypeTok{clinton_votes =} \StringTok{`}\DataTypeTok{Hillary Clinton}\StringTok{`}\NormalTok{, }\DataTypeTok{johnson_votes =} \StringTok{`}\DataTypeTok{Gary Johnson}\StringTok{`}\NormalTok{, }\DataTypeTok{total_election =}\NormalTok{ total))}
\end{Highlighting}
\end{Shaded}

\hypertarget{variable-description}{%
\subsubsection{Variable description}\label{variable-description}}

\textbf{precinct:} The name of the precinct.\\
\textbf{trump\_votes:} The number of votes for Donald Trump in that
precinct.\\
\textbf{clinton\_votes:} The number of votes for Hillary Clinton in that
precinct.\\
\textbf{johnson\_votes:} The number of votes for Gary Johnson in that
precinct. \textbf{total\_election:} The number of voters who
participated in the 2016 election.

\hypertarget{data-summary-1}{%
\subsubsection{Data summary}\label{data-summary-1}}

\begin{Shaded}
\begin{Highlighting}[]
\KeywordTok{View}\NormalTok{(election_tidy)}
\KeywordTok{export_summary_table}\NormalTok{(}\KeywordTok{dfSummary}\NormalTok{(election_tidy))}
\end{Highlighting}
\end{Shaded}

\hypertarget{gwinnett-transportation-expansion-plan-map}{%
\subsection{Gwinnett Transportation Expansion Plan
Map}\label{gwinnett-transportation-expansion-plan-map}}

\textbf{Citation:} Plan Documents \textbar{} Gwinnett County. (n.d.).
Retrieved March 28, 2020, from
\url{https://www.gwinnettcounty.com/web/gwinnett/departments/transportation/connectgwinnett/plandocuments}\\
\textbf{DOI:} N/A\\
\textbf{Date Downloaded:} Mar 28, 2020\\
\textbf{Filenames:} raw\_data/existing\_map.pdf,
raw\_data/short\_range\_map.pdf, raw\_data/mid\_range\_map.pdf,
raw\_data/long\_range\_1\_map.pdf, raw\_data/long\_range\_2\_map.pdf\\
\textbf{Unit of observation:} N/A\\
\textbf{Dates covered:} N/A

\hypertarget{to-obtain-a-copy-3}{%
\subsubsection{To obtain a copy}\label{to-obtain-a-copy-3}}

A copy of the maps can be obtained at the official website of Gwinnett
County with the following steps:

\begin{enumerate}
\def\labelenumi{\arabic{enumi}.}
\tightlist
\item
  Go to \url{https://www.gwinnettcounty.com/web/gwinnett/Home}.
\item
  Select \textbf{Transportation} under the \textbf{Departments} tab.
\item
  Click on \textbf{Connect Gwinnett: Transit Plan} on the left panel.
\item
  Click on \textbf{Plan Documents} on the left panel.
\item
  Under the Reports title, there are detailed descriptions of the
  transit plan from short-term to long-term period. Under the
  \textbf{Maps} title, users can find the maps that are used in this
  research.
\item
  Click on the map that is interested and save the PDF file to get a
  copy.
\end{enumerate}

\hypertarget{data-processing-and-combination}{%
\section{Data processing and
combination}\label{data-processing-and-combination}}

\hypertarget{referendum-and-election-data-processing}{%
\subsection{Referendum and election data
processing}\label{referendum-and-election-data-processing}}

Firstly, select the desired variables in the referendum dataset:
precinct name, total votes, total registered voters, and the number of
supporters.

\begin{Shaded}
\begin{Highlighting}[]
\NormalTok{vote_tidy <-}\StringTok{ }\NormalTok{vote_result }\OperatorTok
\StringTok{  }\KeywordTok{select}\NormalTok{(precinct, total_votes, registered_voters, votes_yes)}
\end{Highlighting}
\end{Shaded}

Next, select the variables in the 2016 election dataset: precinct name,
total votes, and the number of Trump supporters. Then join the two
tables by precinct name.

\begin{Shaded}
\begin{Highlighting}[]
\NormalTok{precinct_tidy <-}\StringTok{ }\NormalTok{election_tidy }\OperatorTok
\StringTok{  }\KeywordTok{select}\NormalTok{(precinct, trump_votes, total_election) }\OperatorTok
\StringTok{  }\KeywordTok{full_join}\NormalTok{(vote_tidy, }\DataTypeTok{by =} \StringTok{"precinct"}\NormalTok{)}
\end{Highlighting}
\end{Shaded}

The datasets above are measured at the precinct level. To match with
census data, they will be transformed into census tracts by using
shapefiles. Additionally, the shapefile contains a typo in the precinct
name, and is fixed in the code trunk below.

\begin{Shaded}
\begin{Highlighting}[]
\NormalTok{precincts <-}\StringTok{ }\KeywordTok{st_read}\NormalTok{(}\StringTok{"raw_data/VTD2018-Shapefile.shp"}\NormalTok{)}
\end{Highlighting}
\end{Shaded}

Reading layer
\texttt{VTD2018-Shapefile\textquotesingle{}\ from\ data\ source}C:\Users\iris\_\OneDrive\Desktop\Smith\Spring 2020\ECO 324\seminar-paper-Shengqi-Iris-Zhong\raw\_data\VTD2018-Shapefile.shp'
using driver `ESRI Shapefile' Simple feature collection with 2660
features and 11 fields geometry type: MULTIPOLYGON dimension: XY bbox:
xmin: -85.60516 ymin: 30.35576 xmax: -80.75143 ymax: 35.00066
proj4string: +proj=longlat +ellps=GRS80 +no\_defs

\begin{Shaded}
\begin{Highlighting}[]
\NormalTok{gwinnett <-}\StringTok{ }\NormalTok{precincts }\OperatorTok\StringTok{ }
\StringTok{  }\KeywordTok{filter}\NormalTok{(CTYNAME }\OperatorTok{==}\StringTok{ "GWINNETT"}\NormalTok{) }\OperatorTok\StringTok{ }
\StringTok{  }\KeywordTok{st_transform}\NormalTok{(}\StringTok{"+init=epsg:4326"}\NormalTok{) }\OperatorTok\StringTok{  }\CommentTok{# set CRS to WGS84}
\StringTok{  }\KeywordTok{st_transform}\NormalTok{(}\DataTypeTok{crs =} \DecValTok{4269}\NormalTok{)}

\NormalTok{acs_vars <-}\StringTok{ }\KeywordTok{load_variables}\NormalTok{(}\DecValTok{2010}\NormalTok{, }\StringTok{"sf1"}\NormalTok{, }\DataTypeTok{cache =}\NormalTok{ T)}
\NormalTok{gwinnett_tracks <-}\StringTok{ }\NormalTok{tigris}\OperatorTok{::}\KeywordTok{tracts}\NormalTok{(}\DataTypeTok{state =} \DecValTok{13}\NormalTok{, }\DataTypeTok{county =} \DecValTok{135}\NormalTok{) }\OperatorTok\StringTok{ }\KeywordTok{st_as_sf}\NormalTok{()}
\end{Highlighting}
\end{Shaded}

\textbar{} \textbar{} \textbar{} 0\% \textbar{} \textbar{} \textbar{}
1\% \textbar{} \textbar= \textbar{} 1\% \textbar{} \textbar= \textbar{}
2\% \textbar{} \textbar== \textbar{} 2\% \textbar{} \textbar==
\textbar{} 3\% \textbar{} \textbar=== \textbar{} 4\% \textbar{}
\textbar=== \textbar{} 5\% \textbar{} \textbar==== \textbar{} 5\%
\textbar{} \textbar==== \textbar{} 6\% \textbar{} \textbar=====
\textbar{} 7\% \textbar{} \textbar===== \textbar{} 8\% \textbar{}
\textbar====== \textbar{} 8\% \textbar{} \textbar====== \textbar{} 9\%
\textbar{} \textbar======= \textbar{} 9\% \textbar{} \textbar=======
\textbar{} 10\% \textbar{} \textbar======= \textbar{} 11\% \textbar{}
\textbar======== \textbar{} 11\% \textbar{} \textbar======== \textbar{}
12\% \textbar{} \textbar========= \textbar{} 12\% \textbar{}
\textbar========= \textbar{} 13\% \textbar{} \textbar==========
\textbar{} 14\% \textbar{} \textbar========== \textbar{} 15\% \textbar{}
\textbar=========== \textbar{} 15\% \textbar{} \textbar===========
\textbar{} 16\% \textbar{} \textbar============ \textbar{} 17\%
\textbar{} \textbar============ \textbar{} 18\% \textbar{}
\textbar============= \textbar{} 18\% \textbar{} \textbar=============
\textbar{} 19\% \textbar{} \textbar============== \textbar{} 19\%
\textbar{} \textbar============== \textbar{} 20\% \textbar{}
\textbar============== \textbar{} 21\% \textbar{}
\textbar=============== \textbar{} 21\% \textbar{}
\textbar=============== \textbar{} 22\% \textbar{}
\textbar================ \textbar{} 22\% \textbar{}
\textbar================ \textbar{} 23\% \textbar{}
\textbar================ \textbar{} 24\% \textbar{}
\textbar================= \textbar{} 24\% \textbar{}
\textbar================= \textbar{} 25\% \textbar{}
\textbar================== \textbar{} 25\% \textbar{}
\textbar================== \textbar{} 26\% \textbar{}
\textbar=================== \textbar{} 26\% \textbar{}
\textbar=================== \textbar{} 27\% \textbar{}
\textbar=================== \textbar{} 28\% \textbar{}
\textbar==================== \textbar{} 28\% \textbar{}
\textbar==================== \textbar{} 29\% \textbar{}
\textbar===================== \textbar{} 29\% \textbar{}
\textbar===================== \textbar{} 30\% \textbar{}
\textbar===================== \textbar{} 31\% \textbar{}
\textbar====================== \textbar{} 31\% \textbar{}
\textbar====================== \textbar{} 32\% \textbar{}
\textbar======================= \textbar{} 32\% \textbar{}
\textbar======================= \textbar{} 33\% \textbar{}
\textbar======================= \textbar{} 34\% \textbar{}
\textbar======================== \textbar{} 34\% \textbar{}
\textbar======================== \textbar{} 35\% \textbar{}
\textbar========================= \textbar{} 35\% \textbar{}
\textbar========================= \textbar{} 36\% \textbar{}
\textbar========================== \textbar{} 36\% \textbar{}
\textbar========================== \textbar{} 37\% \textbar{}
\textbar========================== \textbar{} 38\% \textbar{}
\textbar=========================== \textbar{} 38\% \textbar{}
\textbar=========================== \textbar{} 39\% \textbar{}
\textbar============================ \textbar{} 39\% \textbar{}
\textbar============================ \textbar{} 40\% \textbar{}
\textbar============================ \textbar{} 41\% \textbar{}
\textbar============================= \textbar{} 41\% \textbar{}
\textbar============================= \textbar{} 42\% \textbar{}
\textbar============================== \textbar{} 42\% \textbar{}
\textbar============================== \textbar{} 43\% \textbar{}
\textbar=============================== \textbar{} 44\% \textbar{}
\textbar=============================== \textbar{} 45\% \textbar{}
\textbar================================ \textbar{} 45\% \textbar{}
\textbar================================ \textbar{} 46\% \textbar{}
\textbar================================= \textbar{} 46\% \textbar{}
\textbar================================= \textbar{} 47\% \textbar{}
\textbar================================= \textbar{} 48\% \textbar{}
\textbar================================== \textbar{} 48\% \textbar{}
\textbar================================== \textbar{} 49\% \textbar{}
\textbar=================================== \textbar{} 49\% \textbar{}
\textbar=================================== \textbar{} 50\% \textbar{}
\textbar=================================== \textbar{} 51\% \textbar{}
\textbar==================================== \textbar{} 51\% \textbar{}
\textbar==================================== \textbar{} 52\% \textbar{}
\textbar===================================== \textbar{} 52\% \textbar{}
\textbar===================================== \textbar{} 53\% \textbar{}
\textbar====================================== \textbar{} 54\%
\textbar{} \textbar====================================== \textbar{}
55\% \textbar{} \textbar=======================================
\textbar{} 55\% \textbar{}
\textbar======================================= \textbar{} 56\%
\textbar{} \textbar======================================== \textbar{}
56\% \textbar{} \textbar========================================
\textbar{} 57\% \textbar{}
\textbar======================================== \textbar{} 58\%
\textbar{} \textbar========================================= \textbar{}
58\% \textbar{} \textbar=========================================
\textbar{} 59\% \textbar{}
\textbar========================================== \textbar{} 59\%
\textbar{} \textbar========================================== \textbar{}
60\% \textbar{} \textbar==========================================
\textbar{} 61\% \textbar{}
\textbar=========================================== \textbar{} 61\%
\textbar{} \textbar===========================================
\textbar{} 62\% \textbar{}
\textbar============================================ \textbar{} 62\%
\textbar{} \textbar============================================
\textbar{} 63\% \textbar{}
\textbar============================================ \textbar{} 64\%
\textbar{} \textbar=============================================
\textbar{} 64\% \textbar{}
\textbar============================================= \textbar{} 65\%
\textbar{} \textbar==============================================
\textbar{} 65\% \textbar{}
\textbar============================================== \textbar{} 66\%
\textbar{} \textbar===============================================
\textbar{} 66\% \textbar{}
\textbar=============================================== \textbar{} 67\%
\textbar{} \textbar===============================================
\textbar{} 68\% \textbar{}
\textbar================================================ \textbar{} 68\%
\textbar{} \textbar================================================
\textbar{} 69\% \textbar{}
\textbar================================================= \textbar{}
69\% \textbar{}
\textbar================================================= \textbar{}
70\% \textbar{}
\textbar================================================= \textbar{}
71\% \textbar{}
\textbar================================================== \textbar{}
71\% \textbar{}
\textbar================================================== \textbar{}
72\% \textbar{}
\textbar=================================================== \textbar{}
72\% \textbar{}
\textbar=================================================== \textbar{}
73\% \textbar{}
\textbar=================================================== \textbar{}
74\% \textbar{}
\textbar==================================================== \textbar{}
74\% \textbar{}
\textbar==================================================== \textbar{}
75\% \textbar{}
\textbar===================================================== \textbar{}
75\% \textbar{}
\textbar===================================================== \textbar{}
76\% \textbar{}
\textbar======================================================
\textbar{} 77\% \textbar{}
\textbar======================================================
\textbar{} 78\% \textbar{}
\textbar=======================================================
\textbar{} 78\% \textbar{}
\textbar=======================================================
\textbar{} 79\% \textbar{}
\textbar========================================================
\textbar{} 79\% \textbar{}
\textbar========================================================
\textbar{} 80\% \textbar{}
\textbar========================================================
\textbar{} 81\% \textbar{}
\textbar=========================================================
\textbar{} 81\% \textbar{}
\textbar=========================================================
\textbar{} 82\% \textbar{}
\textbar==========================================================
\textbar{} 82\% \textbar{}
\textbar==========================================================
\textbar{} 83\% \textbar{}
\textbar==========================================================
\textbar{} 84\% \textbar{}
\textbar===========================================================
\textbar{} 84\% \textbar{}
\textbar===========================================================
\textbar{} 85\% \textbar{}
\textbar============================================================
\textbar{} 85\% \textbar{}
\textbar============================================================
\textbar{} 86\% \textbar{}
\textbar=============================================================
\textbar{} 86\% \textbar{}
\textbar=============================================================
\textbar{} 87\% \textbar{}
\textbar=============================================================
\textbar{} 88\% \textbar{}
\textbar==============================================================
\textbar{} 88\% \textbar{}
\textbar==============================================================
\textbar{} 89\% \textbar{}
\textbar===============================================================
\textbar{} 89\% \textbar{}
\textbar===============================================================
\textbar{} 90\% \textbar{}
\textbar===============================================================
\textbar{} 91\% \textbar{}
\textbar================================================================
\textbar{} 91\% \textbar{}
\textbar================================================================
\textbar{} 92\% \textbar{}
\textbar=================================================================
\textbar{} 92\% \textbar{}
\textbar=================================================================
\textbar{} 93\% \textbar{}
\textbar=================================================================
\textbar{} 94\% \textbar{}
\textbar==================================================================
\textbar{} 94\% \textbar{}
\textbar==================================================================
\textbar{} 95\% \textbar{}
\textbar===================================================================
\textbar{} 95\% \textbar{}
\textbar===================================================================
\textbar{} 96\% \textbar{}
\textbar====================================================================
\textbar{} 96\% \textbar{}
\textbar====================================================================
\textbar{} 97\% \textbar{}
\textbar====================================================================
\textbar{} 98\% \textbar{}
\textbar=====================================================================
\textbar{} 98\% \textbar{}
\textbar=====================================================================
\textbar{} 99\% \textbar{}
\textbar======================================================================\textbar{}
99\% \textbar{}
\textbar======================================================================\textbar{}
100\%

\begin{Shaded}
\begin{Highlighting}[]
\NormalTok{intersected_areas <-}\StringTok{ }\KeywordTok{st_intersection}\NormalTok{(gwinnett, gwinnett_tracks)}
\end{Highlighting}
\end{Shaded}

\begin{verbatim}
## although coordinates are longitude/latitude, st_intersection assumes that they are planar
\end{verbatim}

\begin{verbatim}
## Warning: attribute variables are assumed to be spatially constant throughout all
## geometries
\end{verbatim}

\begin{Shaded}
\begin{Highlighting}[]
\NormalTok{area_values <-}\StringTok{ }\NormalTok{intersected_areas }\OperatorTok\StringTok{ }\KeywordTok{mutate}\NormalTok{(}\DataTypeTok{loc_area =} \KeywordTok{st_area}\NormalTok{(intersected_areas))}

\NormalTok{precinct_areas <-}\StringTok{ }\NormalTok{area_values }\OperatorTok\StringTok{ }
\StringTok{  }\KeywordTok{group_by}\NormalTok{(PRECINCT_N) }\OperatorTok\StringTok{ }
\StringTok{  }\KeywordTok{summarize}\NormalTok{(}\DataTypeTok{total_area =} \KeywordTok{sum}\NormalTok{(loc_area)) }\OperatorTok\StringTok{ }
\StringTok{  }\KeywordTok{st_drop_geometry}\NormalTok{() }\OperatorTok\StringTok{ }
\StringTok{  }\KeywordTok{right_join}\NormalTok{(area_values) }\OperatorTok\StringTok{ }
\StringTok{  }\KeywordTok{mutate}\NormalTok{(}\DataTypeTok{shr_of_precinct =} \KeywordTok{as.vector}\NormalTok{(loc_area}\OperatorTok{/}\NormalTok{total_area))}\CommentTok{# %>%}
\end{Highlighting}
\end{Shaded}

\begin{verbatim}
## Joining, by = "PRECINCT_N"
\end{verbatim}

\begin{Shaded}
\begin{Highlighting}[]
 \CommentTok{# select(PRECINCT_I, PRECINCT_N, GEOID, shr_of_precinct)}

\NormalTok{vote_allocation_shares <-}\StringTok{ }\KeywordTok{as.data.frame}\NormalTok{(precinct_areas }\OperatorTok\StringTok{ }
\StringTok{  }\KeywordTok{select}\NormalTok{(PRECINCT_I, PRECINCT_N, GEOID, shr_of_precinct) }\OperatorTok\StringTok{ }
\StringTok{  }\KeywordTok{group_by}\NormalTok{(PRECINCT_I, PRECINCT_N, GEOID) }\OperatorTok\StringTok{ }
\StringTok{  }\KeywordTok{summarize}\NormalTok{(}\DataTypeTok{shr_of_precinct =} \KeywordTok{sum}\NormalTok{(shr_of_precinct)) }\OperatorTok\StringTok{ }
\StringTok{  }\KeywordTok{filter}\NormalTok{(shr_of_precinct }\OperatorTok{>}\StringTok{ }\FloatTok{.001}\NormalTok{))}

\NormalTok{vote_allocation_shares}\OperatorTok{$}\NormalTok{PRECINCT_N <-}\StringTok{ }\KeywordTok{as.character}\NormalTok{(vote_allocation_shares}\OperatorTok{$}\NormalTok{PRECINCT_N)}
\NormalTok{vote_allocation_shares}\OperatorTok{$}\NormalTok{PRECINCT_N[}\DecValTok{35}\NormalTok{]=}\StringTok{ "PINCKNEYVILLE A"}
\NormalTok{vote_allocation_shares}\OperatorTok{$}\NormalTok{PRECINCT_N[}\DecValTok{36}\NormalTok{]=}\StringTok{ "PINCKNEYVILLE A"}
\NormalTok{vote_allocation_shares}\OperatorTok{$}\NormalTok{PRECINCT_N <-}\StringTok{ }\KeywordTok{as.factor}\NormalTok{(vote_allocation_shares}\OperatorTok{$}\NormalTok{PRECINCT_N)}
\end{Highlighting}
\end{Shaded}

\begin{Shaded}
\begin{Highlighting}[]
\NormalTok{precinct_tidy_transformed <-}\StringTok{ }\NormalTok{vote_allocation_shares }\OperatorTok
\StringTok{  }\KeywordTok{left_join}\NormalTok{(precinct_tidy, }\DataTypeTok{by =} \KeywordTok{c}\NormalTok{(}\StringTok{"PRECINCT_N"}\NormalTok{ =}\StringTok{ "precinct"}\NormalTok{)) }\OperatorTok
\StringTok{  }\KeywordTok{mutate}\NormalTok{(}\DataTypeTok{supporter_share =}\NormalTok{ votes_yes }\OperatorTok{*}\StringTok{ }\NormalTok{shr_of_precinct,}
         \DataTypeTok{voter_share =}\NormalTok{ total_votes }\OperatorTok{*}\StringTok{ }\NormalTok{shr_of_precinct,}
         \DataTypeTok{total_voter_share =}\NormalTok{ registered_voters }\OperatorTok{*}\StringTok{ }\NormalTok{shr_of_precinct,}
         \DataTypeTok{trump_share =}\NormalTok{ trump_votes }\OperatorTok{*}\StringTok{ }\NormalTok{shr_of_precinct,}
         \DataTypeTok{election_voter_share =}\NormalTok{ total_election }\OperatorTok{*}\StringTok{ }\NormalTok{shr_of_precinct}
\NormalTok{         ) }\OperatorTok
\StringTok{  }\KeywordTok{group_by}\NormalTok{(GEOID) }\OperatorTok
\StringTok{  }\KeywordTok{summarize}\NormalTok{(}\DataTypeTok{trump_pct =} \KeywordTok{sum}\NormalTok{(trump_share) }\OperatorTok{/}\StringTok{ }\KeywordTok{sum}\NormalTok{(election_voter_share),}
            \DataTypeTok{voter_turnout =} \KeywordTok{sum}\NormalTok{(voter_share) }\OperatorTok{/}\StringTok{ }\KeywordTok{sum}\NormalTok{(total_voter_share),}
            \DataTypeTok{yes_pct =} \KeywordTok{sum}\NormalTok{(supporter_share) }\OperatorTok{/}\StringTok{ }\KeywordTok{sum}\NormalTok{(voter_share))}
\end{Highlighting}
\end{Shaded}

\begin{verbatim}
## Warning: Column `PRECINCT_N`/`precinct` joining factor and character vector,
## coercing into character vector
\end{verbatim}

\hypertarget{map-data-processing}{%
\subsection{Map data processing}\label{map-data-processing}}

The PDF versions of current and short-range plan (Y2020-2025) maps have
been transformed to shapefiles through QGIS. The following code chunk
reads the shapefile and creates a data-frame that specifies whether each
transportation line is within 500 meters of distance in that census
tract.

\begin{Shaded}
\begin{Highlighting}[]
\NormalTok{short_range <-}\StringTok{ }\KeywordTok{st_read}\NormalTok{(}\StringTok{"raw_data/short_range_layer.shp"}\NormalTok{) }\OperatorTok\StringTok{ }
\StringTok{  }\KeywordTok{st_transform}\NormalTok{(}\DecValTok{32615}\NormalTok{)}
\end{Highlighting}
\end{Shaded}

Reading layer
\texttt{short\_range\_layer\textquotesingle{}\ from\ data\ source}C:\Users\iris\_\OneDrive\Desktop\Smith\Spring 2020\ECO 324\seminar-paper-Shengqi-Iris-Zhong\raw\_data\short\_range\_layer.shp'
using driver `ESRI Shapefile' Simple feature collection with 26 features
and 2 fields geometry type: LINESTRING dimension: XY bbox: xmin:
-84.42795 ymin: 33.72982 xmax: -83.89378 ymax: 34.08735 CRS: 4326

\begin{Shaded}
\begin{Highlighting}[]
\NormalTok{short_range_buffers <-}\StringTok{ }\KeywordTok{st_buffer}\NormalTok{(short_range, }\DataTypeTok{dist =} \DecValTok{500}\NormalTok{)}
\NormalTok{acs_vars <-}\StringTok{ }\KeywordTok{load_variables}\NormalTok{(}\DecValTok{2010}\NormalTok{, }\StringTok{"sf1"}\NormalTok{, }\DataTypeTok{cache =}\NormalTok{ T)}
\NormalTok{gwinnett_tracks <-}\StringTok{ }\NormalTok{tigris}\OperatorTok{::}\KeywordTok{tracts}\NormalTok{(}\DataTypeTok{state =} \DecValTok{13}\NormalTok{, }\DataTypeTok{county =} \DecValTok{135}\NormalTok{, }\DataTypeTok{cb =}\NormalTok{ T) }\OperatorTok\StringTok{ }\KeywordTok{st_as_sf}\NormalTok{() }\OperatorTok\StringTok{ }
\StringTok{  }\KeywordTok{st_transform}\NormalTok{(}\DecValTok{32615}\NormalTok{)}
\end{Highlighting}
\end{Shaded}

\textbar{} \textbar{} \textbar{} 0\% \textbar{} \textbar= \textbar{} 1\%
\textbar{} \textbar= \textbar{} 2\% \textbar{} \textbar== \textbar{} 3\%
\textbar{} \textbar== \textbar{} 4\% \textbar{} \textbar=== \textbar{}
4\% \textbar{} \textbar==== \textbar{} 5\% \textbar{} \textbar====
\textbar{} 6\% \textbar{} \textbar===== \textbar{} 7\% \textbar{}
\textbar====== \textbar{} 8\% \textbar{} \textbar====== \textbar{} 9\%
\textbar{} \textbar======= \textbar{} 9\% \textbar{} \textbar=======
\textbar{} 10\% \textbar{} \textbar======== \textbar{} 11\% \textbar{}
\textbar======== \textbar{} 12\% \textbar{} \textbar========= \textbar{}
13\% \textbar{} \textbar========== \textbar{} 14\% \textbar{}
\textbar=========== \textbar{} 15\% \textbar{} \textbar===========
\textbar{} 16\% \textbar{} \textbar============ \textbar{} 17\%
\textbar{} \textbar============ \textbar{} 18\% \textbar{}
\textbar============= \textbar{} 18\% \textbar{} \textbar=============
\textbar{} 19\% \textbar{} \textbar============== \textbar{} 20\%
\textbar{} \textbar=============== \textbar{} 21\% \textbar{}
\textbar=============== \textbar{} 22\% \textbar{}
\textbar================ \textbar{} 22\% \textbar{}
\textbar================ \textbar{} 23\% \textbar{}
\textbar================= \textbar{} 24\% \textbar{}
\textbar================= \textbar{} 25\% \textbar{}
\textbar================== \textbar{} 25\% \textbar{}
\textbar================== \textbar{} 26\% \textbar{}
\textbar=================== \textbar{} 27\% \textbar{}
\textbar==================== \textbar{} 28\% \textbar{}
\textbar==================== \textbar{} 29\% \textbar{}
\textbar===================== \textbar{} 30\% \textbar{}
\textbar===================== \textbar{} 31\% \textbar{}
\textbar====================== \textbar{} 31\% \textbar{}
\textbar====================== \textbar{} 32\% \textbar{}
\textbar======================= \textbar{} 32\% \textbar{}
\textbar======================= \textbar{} 33\% \textbar{}
\textbar======================== \textbar{} 34\% \textbar{}
\textbar======================== \textbar{} 35\% \textbar{}
\textbar========================= \textbar{} 36\% \textbar{}
\textbar========================== \textbar{} 37\% \textbar{}
\textbar=========================== \textbar{} 38\% \textbar{}
\textbar=========================== \textbar{} 39\% \textbar{}
\textbar============================ \textbar{} 40\% \textbar{}
\textbar============================ \textbar{} 41\% \textbar{}
\textbar============================= \textbar{} 41\% \textbar{}
\textbar============================= \textbar{} 42\% \textbar{}
\textbar============================== \textbar{} 43\% \textbar{}
\textbar=============================== \textbar{} 44\% \textbar{}
\textbar=============================== \textbar{} 45\% \textbar{}
\textbar================================ \textbar{} 45\% \textbar{}
\textbar================================ \textbar{} 46\% \textbar{}
\textbar================================= \textbar{} 47\% \textbar{}
\textbar================================= \textbar{} 48\% \textbar{}
\textbar================================== \textbar{} 49\% \textbar{}
\textbar=================================== \textbar{} 50\% \textbar{}
\textbar==================================== \textbar{} 51\% \textbar{}
\textbar==================================== \textbar{} 52\% \textbar{}
\textbar===================================== \textbar{} 53\% \textbar{}
\textbar====================================== \textbar{} 54\%
\textbar{} \textbar====================================== \textbar{}
55\% \textbar{} \textbar=======================================
\textbar{} 56\% \textbar{}
\textbar======================================== \textbar{} 57\%
\textbar{} \textbar======================================== \textbar{}
58\% \textbar{} \textbar=========================================
\textbar{} 58\% \textbar{}
\textbar========================================== \textbar{} 59\%
\textbar{} \textbar========================================== \textbar{}
60\% \textbar{} \textbar==========================================
\textbar{} 61\% \textbar{}
\textbar=========================================== \textbar{} 61\%
\textbar{} \textbar===========================================
\textbar{} 62\% \textbar{}
\textbar============================================ \textbar{} 63\%
\textbar{} \textbar=============================================
\textbar{} 64\% \textbar{}
\textbar============================================= \textbar{} 65\%
\textbar{} \textbar==============================================
\textbar{} 66\% \textbar{}
\textbar=============================================== \textbar{} 67\%
\textbar{} \textbar================================================
\textbar{} 68\% \textbar{}
\textbar================================================ \textbar{} 69\%
\textbar{} \textbar=================================================
\textbar{} 70\% \textbar{}
\textbar================================================= \textbar{}
71\% \textbar{}
\textbar================================================== \textbar{}
72\% \textbar{}
\textbar=================================================== \textbar{}
73\% \textbar{}
\textbar==================================================== \textbar{}
74\% \textbar{}
\textbar===================================================== \textbar{}
75\% \textbar{}
\textbar===================================================== \textbar{}
76\% \textbar{}
\textbar======================================================
\textbar{} 76\% \textbar{}
\textbar======================================================
\textbar{} 77\% \textbar{}
\textbar======================================================
\textbar{} 78\% \textbar{}
\textbar=======================================================
\textbar{} 79\% \textbar{}
\textbar========================================================
\textbar{} 80\% \textbar{}
\textbar=========================================================
\textbar{} 81\% \textbar{}
\textbar==========================================================
\textbar{} 82\% \textbar{}
\textbar==========================================================
\textbar{} 83\% \textbar{}
\textbar===========================================================
\textbar{} 84\% \textbar{}
\textbar============================================================
\textbar{} 85\% \textbar{}
\textbar============================================================
\textbar{} 86\% \textbar{}
\textbar=============================================================
\textbar{} 87\% \textbar{}
\textbar==============================================================
\textbar{} 88\% \textbar{}
\textbar==============================================================
\textbar{} 89\% \textbar{}
\textbar===============================================================
\textbar{} 90\% \textbar{}
\textbar================================================================
\textbar{} 91\% \textbar{}
\textbar================================================================
\textbar{} 92\% \textbar{}
\textbar=================================================================
\textbar{} 92\% \textbar{}
\textbar=================================================================
\textbar{} 93\% \textbar{}
\textbar==================================================================
\textbar{} 94\% \textbar{}
\textbar===================================================================
\textbar{} 95\% \textbar{}
\textbar===================================================================
\textbar{} 96\% \textbar{}
\textbar====================================================================
\textbar{} 97\% \textbar{}
\textbar=====================================================================
\textbar{} 98\% \textbar{}
\textbar=====================================================================
\textbar{} 99\% \textbar{}
\textbar======================================================================\textbar{}
99\% \textbar{}
\textbar======================================================================\textbar{}
100\%

\begin{Shaded}
\begin{Highlighting}[]
\NormalTok{short_in_buffer <-}\StringTok{ }\KeywordTok{st_intersects}\NormalTok{(gwinnett_tracks, short_range_buffers, }\DataTypeTok{sparse =}\NormalTok{ F)}
\KeywordTok{colnames}\NormalTok{(short_in_buffer) <-}\StringTok{ }\NormalTok{short_range_buffers}\OperatorTok{$}\NormalTok{id}
\NormalTok{short_in_buffer <-}\StringTok{ }\KeywordTok{as_tibble}\NormalTok{(short_in_buffer)}
\NormalTok{short_in_buffer}\OperatorTok{$}\NormalTok{GEOID <-}\StringTok{ }\NormalTok{gwinnett_tracks}\OperatorTok{$}\NormalTok{GEOID}
\end{Highlighting}
\end{Shaded}

Then, I summarize whether any transportation line is within 500 meters
of distance in that tract.

\begin{Shaded}
\begin{Highlighting}[]
\NormalTok{short_in_buffer <-}\StringTok{ }\NormalTok{short_in_buffer }\OperatorTok
\StringTok{  }\KeywordTok{mutate}\NormalTok{(}\DataTypeTok{plan_yes =} \KeywordTok{case_when}\NormalTok{(}\KeywordTok{rowSums}\NormalTok{(short_in_buffer[,}\KeywordTok{c}\NormalTok{(}\StringTok{"1"}\NormalTok{, }\StringTok{"2"}\NormalTok{,}\StringTok{"3"}\NormalTok{, }\StringTok{"4"}\NormalTok{,}\StringTok{"5"}\NormalTok{, }\StringTok{"6"}\NormalTok{,}\StringTok{"7"}\NormalTok{, }\StringTok{"8"}\NormalTok{,}\StringTok{"9"}\NormalTok{, }\StringTok{"10"}\NormalTok{,}\StringTok{"11"}\NormalTok{, }\StringTok{"12"}\NormalTok{,}\StringTok{"13"}\NormalTok{, }\StringTok{"14"}\NormalTok{,}\StringTok{"15"}\NormalTok{, }\StringTok{"16"}\NormalTok{,}\StringTok{"17"}\NormalTok{, }\StringTok{"18"}\NormalTok{,}\StringTok{"19"}\NormalTok{, }\StringTok{"20"}\NormalTok{,}\StringTok{"21"}\NormalTok{, }\StringTok{"22"}\NormalTok{,}\StringTok{"23"}\NormalTok{,}\StringTok{"24"}\NormalTok{,}\StringTok{"25"}\NormalTok{, }\StringTok{"26"}\NormalTok{)])}\OperatorTok{!=}\DecValTok{0} \OperatorTok{~}\StringTok{ }\DecValTok{1}\NormalTok{,}
         \KeywordTok{rowSums}\NormalTok{(short_in_buffer[,}\KeywordTok{c}\NormalTok{(}\StringTok{"1"}\NormalTok{, }\StringTok{"2"}\NormalTok{,}\StringTok{"3"}\NormalTok{, }\StringTok{"4"}\NormalTok{,}\StringTok{"5"}\NormalTok{, }\StringTok{"6"}\NormalTok{,}\StringTok{"7"}\NormalTok{, }\StringTok{"8"}\NormalTok{,}\StringTok{"9"}\NormalTok{, }\StringTok{"10"}\NormalTok{,}\StringTok{"11"}\NormalTok{, }\StringTok{"12"}\NormalTok{,}\StringTok{"13"}\NormalTok{, }\StringTok{"14"}\NormalTok{,}\StringTok{"15"}\NormalTok{, }\StringTok{"16"}\NormalTok{,}\StringTok{"17"}\NormalTok{, }\StringTok{"18"}\NormalTok{,}\StringTok{"19"}\NormalTok{, }\StringTok{"20"}\NormalTok{,}\StringTok{"21"}\NormalTok{, }\StringTok{"22"}\NormalTok{,}\StringTok{"23"}\NormalTok{,}\StringTok{"24"}\NormalTok{,}\StringTok{"25"}\NormalTok{, }\StringTok{"26"}\NormalTok{)])}\OperatorTok{==}\DecValTok{0} \OperatorTok{~}\StringTok{ }\DecValTok{0}\NormalTok{)) }\OperatorTok
\StringTok{  }\KeywordTok{select}\NormalTok{(GEOID, plan_yes)}
\end{Highlighting}
\end{Shaded}

Similar steps are taken to get the data of current transportation
system.

\begin{Shaded}
\begin{Highlighting}[]
\NormalTok{current <-}\StringTok{ }\KeywordTok{st_read}\NormalTok{(}\StringTok{"raw_data/existing_layer.shp"}\NormalTok{) }\OperatorTok\StringTok{ }
\StringTok{  }\KeywordTok{st_transform}\NormalTok{(}\DecValTok{32615}\NormalTok{)}
\end{Highlighting}
\end{Shaded}

Reading layer
\texttt{existing\_layer\textquotesingle{}\ from\ data\ source}C:\Users\iris\_\OneDrive\Desktop\Smith\Spring 2020\ECO 324\seminar-paper-Shengqi-Iris-Zhong\raw\_data\existing\_layer.shp'
using driver `ESRI Shapefile' Simple feature collection with 12 features
and 2 fields geometry type: LINESTRING dimension: XY bbox: xmin:
-84.4364 ymin: 33.71804 xmax: -83.98294 ymax: 34.08337 CRS: 4326

\begin{Shaded}
\begin{Highlighting}[]
\NormalTok{current_buffers <-}\StringTok{ }\KeywordTok{st_buffer}\NormalTok{(current, }\DataTypeTok{dist =} \DecValTok{500}\NormalTok{)}
\NormalTok{current_in_buffer <-}\StringTok{ }\KeywordTok{st_intersects}\NormalTok{(gwinnett_tracks, current_buffers, }\DataTypeTok{sparse =}\NormalTok{ F)}
\KeywordTok{colnames}\NormalTok{(current_in_buffer) <-}\StringTok{ }\NormalTok{current_buffers}\OperatorTok{$}\NormalTok{id}
\NormalTok{current_in_buffer <-}\StringTok{ }\KeywordTok{as_tibble}\NormalTok{(current_in_buffer)}
\NormalTok{current_in_buffer}\OperatorTok{$}\NormalTok{GEOID <-}\StringTok{ }\NormalTok{gwinnett_tracks}\OperatorTok{$}\NormalTok{GEOID}
\end{Highlighting}
\end{Shaded}

\begin{Shaded}
\begin{Highlighting}[]
\NormalTok{current_in_buffer <-}\StringTok{ }\NormalTok{current_in_buffer }\OperatorTok
\StringTok{  }\KeywordTok{mutate}\NormalTok{(}\DataTypeTok{current_yes =} \KeywordTok{case_when}\NormalTok{(}\KeywordTok{rowSums}\NormalTok{(current_in_buffer[,}\KeywordTok{c}\NormalTok{(}\StringTok{"1"}\NormalTok{, }\StringTok{"2"}\NormalTok{,}\StringTok{"3"}\NormalTok{, }\StringTok{"4"}\NormalTok{,}\StringTok{"5"}\NormalTok{, }\StringTok{"6"}\NormalTok{,}\StringTok{"7"}\NormalTok{, }\StringTok{"8"}\NormalTok{,}\StringTok{"9"}\NormalTok{, }\StringTok{"10"}\NormalTok{,}\StringTok{"11"}\NormalTok{, }\StringTok{"12"}\NormalTok{)])}\OperatorTok{!=}\DecValTok{0} \OperatorTok{~}\StringTok{ }\DecValTok{1}\NormalTok{,}
         \KeywordTok{rowSums}\NormalTok{(current_in_buffer[,}\KeywordTok{c}\NormalTok{(}\StringTok{"1"}\NormalTok{, }\StringTok{"2"}\NormalTok{,}\StringTok{"3"}\NormalTok{, }\StringTok{"4"}\NormalTok{,}\StringTok{"5"}\NormalTok{, }\StringTok{"6"}\NormalTok{,}\StringTok{"7"}\NormalTok{, }\StringTok{"8"}\NormalTok{,}\StringTok{"9"}\NormalTok{, }\StringTok{"10"}\NormalTok{,}\StringTok{"11"}\NormalTok{, }\StringTok{"12"}\NormalTok{)])}\OperatorTok{==}\DecValTok{0} \OperatorTok{~}\StringTok{ }\DecValTok{0}\NormalTok{)) }\OperatorTok
\StringTok{  }\KeywordTok{select}\NormalTok{(GEOID, current_yes)}
\end{Highlighting}
\end{Shaded}

\hypertarget{census-data-processing}{%
\subsection{Census data processing}\label{census-data-processing}}

Select the variables in census data: GEOID, median age, median income,
the percentage of white, the percentage of people using public
transportation, and the percentage of people who spend more than an hour
on transportation to work. Finally, join the tables together to get the
final dataset.

\begin{Shaded}
\begin{Highlighting}[]
\NormalTok{final_data <-}\StringTok{ }\NormalTok{cbdata_tidy }\OperatorTok
\StringTok{  }\KeywordTok{mutate}\NormalTok{(}\DataTypeTok{time_pct =}\NormalTok{ time_}\DecValTok{60}\NormalTok{_}\DecValTok{89}\NormalTok{_pct }\OperatorTok{+}\StringTok{ }\NormalTok{time_more_}\DecValTok{90}\NormalTok{_pct) }\OperatorTok
\StringTok{  }\KeywordTok{select}\NormalTok{(GEOID, medage, medincome, white_pct, public_pct, time_pct) }\OperatorTok
\StringTok{  }\KeywordTok{inner_join}\NormalTok{(precinct_tidy_transformed, }\DataTypeTok{by =} \StringTok{"GEOID"}\NormalTok{) }\OperatorTok
\StringTok{  }\KeywordTok{inner_join}\NormalTok{(short_in_buffer, }\DataTypeTok{by =} \StringTok{"GEOID"}\NormalTok{) }\OperatorTok
\StringTok{  }\KeywordTok{inner_join}\NormalTok{(current_in_buffer, }\DataTypeTok{by =} \StringTok{"GEOID"}\NormalTok{) }\OperatorTok
\StringTok{  }\KeywordTok{mutate}\NormalTok{(}\DataTypeTok{plan_yes =} \KeywordTok{as.factor}\NormalTok{(plan_yes),}
         \DataTypeTok{current_yes =} \KeywordTok{as.factor}\NormalTok{(current_yes))}
\end{Highlighting}
\end{Shaded}

\hypertarget{analysis-variables}{%
\section{Analysis Variables}\label{analysis-variables}}

The variable used in the final analysis are:

\begin{itemize}
\tightlist
\item
  \textbf{GEOID:} The geographic identifier of the census tract.\\
\item
  \textbf{medage:} The median age of the population in the tract.
\item
  \textbf{medincome:} The median income of the population in the tract.
\item
  \textbf{white\_pct:} The percentage of white population in the tract.
\item
  \textbf{public\_pct:} The percentage of people who go to work by
  public transportation (excluding taxi or cab).\\
\item
  \textbf{time\_pct:} The percentage of people who travel more than an
  hour to work.\\
\item
  \textbf{trump\_pct:} The estimated percentage of votes for Donald
  Trump in that tract.
\item
  \textbf{voter\_turnout:} The estimated percentage of voters who voted
  in this referendum in the tract.
\item
  \textbf{yes\_pct:} The estimated percentage of voters who voted yes in
  this referendum in the tract.
\item
  \textbf{plan\_yes:} Whether the tract is covered by the public
  transportation planned in the short-range (Y2020 -- 2025), defined by
  whether any transportation is available within 500 meters. 1 stands
  for yes, 0 stands for no.\\
\item
  \textbf{current\_yes:} Whether the tract is covered by the existing
  public transportation, defined by whether any transportation is
  available within 500 meters. 1 stands for yes, 0 stands for no.
\end{itemize}

\begin{Shaded}
\begin{Highlighting}[]
\KeywordTok{export_summary_table}\NormalTok{(}\KeywordTok{dfSummary}\NormalTok{(final_data))}
\KeywordTok{save}\NormalTok{(}\StringTok{"final_data"}\NormalTok{, }\DataTypeTok{file =} \StringTok{"processed_data/analysis_data.RData"}\NormalTok{)}
\end{Highlighting}
\end{Shaded}

\hypertarget{discussion-of-data}{%
\section{Discussion of Data}\label{discussion-of-data}}

First, the distribution of median income is slightly skewed to the
right, ranging from 33020 to 156136. It demonstrates that income
inequality is visible at an aggregate level.

Secondly, the usage of public transit is surprisingly low. The median is
0, implying that people in over half of the tracts do not use public
transport. The distribution is highly positively skewed and might
require data transformation.

The percentage of Trump supporters is relatively evenly distributed,
with a mean of 0.4. Given that Gwinnett is one of the minority counties
in which Clinton won over Trump, the result is reasonable.

Finally, as expected, public transportation is accessible to more tracts
if it expands as planned. Interestingly, through a more detailed look at
the data, we can see a few census tracts covered by public transit
within 500 meters of distance at present, but no more in the short-range
plan. This is possibly due to a reduction in circuitous routing in the
proposition.

\end{document}
