% Options for packages loaded elsewhere
\PassOptionsToPackage{unicode}{hyperref}
\PassOptionsToPackage{hyphens}{url}
%
\documentclass[
]{article}
\usepackage{lmodern}
\usepackage{amssymb,amsmath}
\usepackage{ifxetex,ifluatex}
\ifnum 0\ifxetex 1\fi\ifluatex 1\fi=0 % if pdftex
  \usepackage[T1]{fontenc}
  \usepackage[utf8]{inputenc}
  \usepackage{textcomp} % provide euro and other symbols
\else % if luatex or xetex
  \usepackage{unicode-math}
  \defaultfontfeatures{Scale=MatchLowercase}
  \defaultfontfeatures[\rmfamily]{Ligatures=TeX,Scale=1}
\fi
% Use upquote if available, for straight quotes in verbatim environments
\IfFileExists{upquote.sty}{\usepackage{upquote}}{}
\IfFileExists{microtype.sty}{% use microtype if available
  \usepackage[]{microtype}
  \UseMicrotypeSet[protrusion]{basicmath} % disable protrusion for tt fonts
}{}
\makeatletter
\@ifundefined{KOMAClassName}{% if non-KOMA class
  \IfFileExists{parskip.sty}{%
    \usepackage{parskip}
  }{% else
    \setlength{\parindent}{0pt}
    \setlength{\parskip}{6pt plus 2pt minus 1pt}}
}{% if KOMA class
  \KOMAoptions{parskip=half}}
\makeatother
\usepackage{xcolor}
\IfFileExists{xurl.sty}{\usepackage{xurl}}{} % add URL line breaks if available
\IfFileExists{bookmark.sty}{\usepackage{bookmark}}{\usepackage{hyperref}}
\hypersetup{
  pdftitle={The Influence of Individual Characterisitcs on Public Transportation Planning},
  pdfauthor={Iris Zhong},
  hidelinks,
  pdfcreator={LaTeX via pandoc}}
\urlstyle{same} % disable monospaced font for URLs
\usepackage[margin=1in]{geometry}
\usepackage{graphicx,grffile}
\makeatletter
\def\maxwidth{\ifdim\Gin@nat@width>\linewidth\linewidth\else\Gin@nat@width\fi}
\def\maxheight{\ifdim\Gin@nat@height>\textheight\textheight\else\Gin@nat@height\fi}
\makeatother
% Scale images if necessary, so that they will not overflow the page
% margins by default, and it is still possible to overwrite the defaults
% using explicit options in \includegraphics[width, height, ...]{}
\setkeys{Gin}{width=\maxwidth,height=\maxheight,keepaspectratio}
% Set default figure placement to htbp
\makeatletter
\def\fps@figure{htbp}
\makeatother
\setlength{\emergencystretch}{3em} % prevent overfull lines
\providecommand{\tightlist}{%
  \setlength{\itemsep}{0pt}\setlength{\parskip}{0pt}}
\setcounter{secnumdepth}{5}
\usepackage{booktabs}
\usepackage{setspace}
\doublespacing
\usepackage[labelfont=bf]{caption}

\title{The Influence of Individual Characterisitcs on Public Transportation
Planning\thanks{xx}}
\author{Iris Zhong}
\date{}

\begin{document}
\maketitle
\begin{abstract}
xx
\end{abstract}

\begin{verbatim}
## Warning: package 'tidyverse' was built under R version 3.5.3
\end{verbatim}

\begin{verbatim}
## Warning: package 'ggplot2' was built under R version 3.5.3
\end{verbatim}

\begin{verbatim}
## Warning: package 'tibble' was built under R version 3.5.3
\end{verbatim}

\begin{verbatim}
## Warning: package 'tidyr' was built under R version 3.5.3
\end{verbatim}

\begin{verbatim}
## Warning: package 'readr' was built under R version 3.5.3
\end{verbatim}

\begin{verbatim}
## Warning: package 'purrr' was built under R version 3.5.3
\end{verbatim}

\begin{verbatim}
## Warning: package 'dplyr' was built under R version 3.5.3
\end{verbatim}

\begin{verbatim}
## Warning: package 'stringr' was built under R version 3.5.3
\end{verbatim}

\begin{verbatim}
## Warning: package 'forcats' was built under R version 3.5.3
\end{verbatim}

\begin{verbatim}
## Warning: package 'lubridate' was built under R version 3.5.3
\end{verbatim}

\begin{verbatim}
## Warning: package 'stargazer' was built under R version 3.5.2
\end{verbatim}

\hypertarget{literature-review}{%
\section{Literature Review}\label{literature-review}}

Allen et al.~(2016) study the reasoning of the failure of a referendum
on a congestion charging scheme in Edinburgh. Instead of using direct
voting data, they conduct a survey after the referendum, which allows
them to ask more specific questions. Researchers can gain detailed data
by surveying, because the unit of measurement is each individual;
however, a possible disadvantage of surveying is that respondents who
turn in the questionnaire tend to have stronger attitudes towards the
proposal, generating sampling bias. They conclude that people who use
cars as the primary transportation mean, demonstrate a misconception of
the pricing plan, or question the effectiveness of the scheme at
reducing congestion are more likely to oppose it. Their findings can
give insights to the similar failure in the Gwinnett referendum. Voters
against the proposal could be those who rarely use public transportation
and those who are not convinced by the effectiveness of expanding public
transit in alleviating the traffic.

Another crucial factor is the accessibility of the proposed transit
system. Kinsey et al.~(2010) examine the relationship between the
distance to the scheduled railway station and voter turnout by studying
the Seattle monorail referendum. They introduce the concept of diffused
and concentrated benefit/cost. People who live far from the monorail
enjoy the diffused benefit of less traffic congestion, and bear the
diffused cost of increased tax. People living close to the rail
experience the same diffused benefit and cost, but they also gain the
concentrated benefit of easily accessing the public good. Finally, those
who live very close to the railway have the same benefits and costs, but
they also face the concentrated cost such as inconvenience during
construction. Since ``people are more strongly motivated to avoid losses
than to approach gains,'' they expect a higher turnout rate in farther
places with votes for ``no,'' which is verified from their analyses.
Besides distance, they also find out precincts with a higher percentage
of people of lower socioeconomic status or young people have a lower
turnout rate. Interestingly, there is a significant interaction between
partisanship and distance, which would be also tested in my study. In
essence, the effect of distance on turnout is weakened by partisanship,
and vanishes beyond a threshold of distance. Even though my dependent
variable is voters' responses rather than turnout, it can be inferred
from Kinsey et al.'s findings that people farther away from the transit
system would vote against the referendum more. However, the relationship
might be non-linear and requires some form of transformation. Regarding
the methods, they utilize the spatial lag model to correct for
autocorrelation, which is proper to use in my project as well since both
studies use precinct-level data.

\end{document}
