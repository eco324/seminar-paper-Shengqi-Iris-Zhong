% Options for packages loaded elsewhere
\PassOptionsToPackage{unicode}{hyperref}
\PassOptionsToPackage{hyphens}{url}
%
\documentclass[
]{article}
\usepackage{lmodern}
\usepackage{amssymb,amsmath}
\usepackage{ifxetex,ifluatex}
\ifnum 0\ifxetex 1\fi\ifluatex 1\fi=0 % if pdftex
  \usepackage[T1]{fontenc}
  \usepackage[utf8]{inputenc}
  \usepackage{textcomp} % provide euro and other symbols
\else % if luatex or xetex
  \usepackage{unicode-math}
  \defaultfontfeatures{Scale=MatchLowercase}
  \defaultfontfeatures[\rmfamily]{Ligatures=TeX,Scale=1}
\fi
% Use upquote if available, for straight quotes in verbatim environments
\IfFileExists{upquote.sty}{\usepackage{upquote}}{}
\IfFileExists{microtype.sty}{% use microtype if available
  \usepackage[]{microtype}
  \UseMicrotypeSet[protrusion]{basicmath} % disable protrusion for tt fonts
}{}
\makeatletter
\@ifundefined{KOMAClassName}{% if non-KOMA class
  \IfFileExists{parskip.sty}{%
    \usepackage{parskip}
  }{% else
    \setlength{\parindent}{0pt}
    \setlength{\parskip}{6pt plus 2pt minus 1pt}}
}{% if KOMA class
  \KOMAoptions{parskip=half}}
\makeatother
\usepackage{xcolor}
\IfFileExists{xurl.sty}{\usepackage{xurl}}{} % add URL line breaks if available
\IfFileExists{bookmark.sty}{\usepackage{bookmark}}{\usepackage{hyperref}}
\hypersetup{
  pdftitle={The Influence of Individual Characterisitcs on Public Transportation Planning},
  pdfauthor={Iris Zhong},
  hidelinks,
  pdfcreator={LaTeX via pandoc}}
\urlstyle{same} % disable monospaced font for URLs
\usepackage[margin=1in]{geometry}
\usepackage{graphicx}
\makeatletter
\def\maxwidth{\ifdim\Gin@nat@width>\linewidth\linewidth\else\Gin@nat@width\fi}
\def\maxheight{\ifdim\Gin@nat@height>\textheight\textheight\else\Gin@nat@height\fi}
\makeatother
% Scale images if necessary, so that they will not overflow the page
% margins by default, and it is still possible to overwrite the defaults
% using explicit options in \includegraphics[width, height, ...]{}
\setkeys{Gin}{width=\maxwidth,height=\maxheight,keepaspectratio}
% Set default figure placement to htbp
\makeatletter
\def\fps@figure{htbp}
\makeatother
\setlength{\emergencystretch}{3em} % prevent overfull lines
\providecommand{\tightlist}{%
  \setlength{\itemsep}{0pt}\setlength{\parskip}{0pt}}
\setcounter{secnumdepth}{5}
\usepackage{booktabs}
\usepackage{setspace}
\doublespacing
\usepackage[labelfont=bf]{caption}
\usepackage{booktabs}
\usepackage{longtable}
\usepackage{array}
\usepackage{multirow}
\usepackage{wrapfig}
\usepackage{float}
\usepackage{colortbl}
\usepackage{pdflscape}
\usepackage{tabu}
\usepackage{threeparttable}
\usepackage{threeparttablex}
\usepackage[normalem]{ulem}
\usepackage{makecell}
\usepackage{xcolor}

\title{The Influence of Individual Characterisitcs on Public
Transportation Planning}
\author{Iris Zhong}
\date{}

\begin{document}
\maketitle
\begin{abstract}
This paper uses a linear regression model to investigate the
sociodemographic factors that shape voters' decisions in a public
transit referendum in Gwinnett County, GA, at an aggregate census tract
level. Partisanship is found to be a significant predictor, such that
the places with a higher proportion of Trump supporters reject the
proposition more. The tracts with a higher concentration of white
population are predicted to have more votes for ``yes'' when other
variables are controlled. Finally, the tracts that do not have access to
public transit at present and will have access to the plan tend to
oppose the referendum. The results suggest the possibility of racism in
white households that they do not favor the idea of using public transit
to connect with neighborhoods with more racial minorities.
\end{abstract}

\hypertarget{introduction}{%
\section{Introduction}\label{introduction}}

Air pollution has become a global concern. Approximately 91\% of the
population around the world live in places where air quality levels
exceed WHO limits in 2016 (Ambient (Outdoor) Air Pollution, n.d.).
Severe air pollution increases the risk for cardiovascular and
respiratory diseases, cancer and adverse birth outcomes, and is
generally associated with higher death rates (WHO \textbar{} Air
Pollution, n.d.). Transportation is a major source of outdoor air
pollution. For example, road transport contributes to up to thirty
percent of particulate emissions in European cities. Developing public
transit is an effective way to reduce air pollution. When people switch
their travel mode from private vehicles to public transport, the
concentration of air pollutants such as carbon monoxide decreases
significantly (Chen \& Whalley, 2012; Zheng et al., 2019).

Constructing a public transit often requires majority approval from the
local voters. Voters either choose yes or no in the ballot, and if more
than half of them vote for yes, the initiative is approved. In public
transit referendums, supporters benefit from public transport by saving
travel time or expenditure, or believe it can relieve traffic congestion
and reduce pollution (Manville \& Cummins, 2015). Meanwhile, opponents
concern with the increase of tax. While some argue that ballot result
does not provide enough information for empirical analyses because it is
a binary response void of the reasoning behind the voter's choice
(Manville \& Levine, 2018), various research studying voting outcomes
has come to insightful conclusions (Burkhardt \& Chan, 2017; Kinsey et
al., 2010, etc.). When enough data is collected and shows patterns of
individuals with similar characteristics making the same decision,
researchers can infer how these factors shape voters' minds. Such an
investigation helps policymakers recognize how different populations
understand public transit and improve the proposition accordingly to
benefit a wider variety of people.

This paper examines a public transit referendum in Gwinnett, Georgia, on
March 23rd, 2019. The referendum is about whether to join Metropolitan
Atlanta Rapid Transit Authority (MARTA) and expand public transit
service. My research asks how sociodemographic characteristics such as
race and partisanship influence voters' decisions. After collecting data
from the government website and census bureau, I convert a few variables
to census tract level, so that they are measured at the same unit. I
create two model specifications, one using the original dataset, the
other using a modified dataset with skewed variables transformed. After
age, commute time, percentage of people using public transit, and voter
turnout are held as controls, partisanship significantly predicts the
percentage of supporters for the referendum -- the tracts with a higher
proportion of Trump supporters tend to vote against the public transit
plan. Race is also a significant predictor, such that the tracts with a
higher proportion of the white population favor the referendum. Finally,
counterintuitively, the tracts newly included in the public transit
system by the proposition tend to reject the plan, even though they gain
more benefit from it.

A possible explanation for my last finding can be related to racial
discrimination, as addressed in popular press articles (Binkovitz,
2017). Gwinnett had already rejected MARTA twice before, mainly because
the white residents in the suburban areas refused to take the same
transport with urban Black people. It is likely that the same reason
applies to the 2019 referendum as well. This speculation seems to
contradict my second finding that white people tend to vote yes in the
referendum. However, it can be argued that a large part of racial
discrimination is explained by the proportion of people supporting
Trump. After holding this variable constant, the relationship between
the proportion of the white population and the percentage of votes for
yes becomes positive.

The remaining paper is structured as follows. Section 2 reviews previous
literature. In Section 3, I include a background of Gwinnett's current
public transit system, the content of the proposition, and the
referendum. In Section 4, I describe the data sources, methods, and
models. Section 5 presents the results and discussions. Finally, Section
6 concludes this paper.

\hypertarget{literature-review}{%
\section{Literature Review}\label{literature-review}}

Referendum data is coded dichotomously with either ``yes'' or ``no.''
Even though it does not seem to provide much information, it is used
intensively in economics research, and its applicability is justified.
For example, Bollino estimates the Italians' WTP of using renewable
energy to produce electricity (2010). He compares the outcome of using
the referendum approach (coding the data into a binary variable) and the
individual stochastic valuation approach. The estimated WTP from the
referendum approach depends on how the original Likert-scale data is
coded into a binary variable, and most of the results are similar to
those from the individual stochastic valuation approach. Even though he
ignores the sociodemographic factors in the referendum approach, the
study proves that using the referendum approach can provide similar
conclusions with other models.

Similarly, Burkhardt and Chan (2017) also utilize referendum data to
estimate WTP for public goods, but at an aggregate zip code level with a
log-likelihood function. They look into thirteen referenda in California
and find out the price of the public good is negatively correlated with
support for the proposition. For the referenda that failed, they
calculate the amount of cost that needs to be cut off for the
proposition to pass by finding the difference between WTP and price.
Even though my project does not involve assessing voters' WTP, this
procedure should be considered in future studies so that an estimation
of the necessary reduction in cost to reach enough support can be
computed. Additionally, I will adopt their method of consolidating
precinct-level data to zip code, in order to transform precinct-level
voting and election datasets to the census tract level.

Allen et al.~(2016) study the reasoning of the failure of a referendum
on a congestion charging scheme in Edinburgh. Instead of using direct
voting data, they conduct a survey after the referendum, which allows
them to ask more specific questions. Researchers can gain detailed data
by surveying, because the unit of measurement is each individual;
however, a possible disadvantage of surveying is that respondents who
turn in the questionnaire tend to have stronger attitudes towards the
proposal, generating sampling bias. They conclude that people who use
cars as the primary transportation mean, demonstrate a misconception of
the pricing plan, or question the effectiveness of the scheme at
reducing congestion are more likely to oppose it. Their findings can
give insights to the similar failure in the Gwinnett referendum. Voters
against the proposal could be those who rarely use public transportation
and those who are not convinced by the effectiveness of expanding public
transit in alleviating the traffic.

Another crucial factor is the accessibility of the proposed transit
system. Kinsey et al.~(2010) examine the relationship between the
distance to the scheduled railway station and voter turnout by studying
the Seattle monorail referendum. They introduce the concept of diffused
and concentrated benefit/cost. People who live far from the monorail
enjoy the diffused benefit of less traffic congestion, and bear the
diffused cost of increased tax. People living close to the rail
experience the same diffused benefit and cost, but they also gain the
concentrated benefit of easily accessing the public good. Finally, those
who live very close to the railway have the same benefits and costs, but
they also face the concentrated cost such as inconvenience during
construction. Since ``people are more strongly motivated to avoid losses
than to approach gains,'' they expect a higher turnout rate in farther
places with votes for ``no,'' which is verified from their analyses.
Besides distance, they also find out precincts with a higher percentage
of people of lower socioeconomic status or young people have a lower
turnout rate. Interestingly, there is a significant interaction between
partisanship and distance, which would be also tested in my study. In
essence, the effect of distance on turnout is weakened by partisanship,
and vanishes beyond a threshold of distance. Even though my dependent
variable is voters' responses rather than turnout, it can be inferred
from Kinsey et al.'s findings that people farther away from the transit
system would vote against the referendum more. However, the relationship
might be non-linear and requires some form of transformation. Regarding
the methods, they utilize the spatial lag model to correct for
autocorrelation, which is proper to use in my project as well since both
studies use precinct-level data.

Finally, socioeconomic factors also play a role in public transit. Ward
discusses the importance of public transit service for the minorities
(2009). Due to the historical influence of redlining and residential
segregation, racial and ethnic minorities often live in ``food deserts
and medically underserved areas.'' They also tend to have lower social
mobility and geographic access when they have low income, less
educational attainment, and no vehicle availability. Therefore, public
transit is essential for them to improve mobility and increase
accessibility. This review supports my hypothesis that racial minorities
are more likely to support this referendum. Besides, I am also concerned
that minorities could have lower participation in the referendum, due to
their lower accessibility to politics, language barriers, etc.

\hypertarget{background}{%
\section{Background}\label{background}}

\hypertarget{gwinnett-county-and-transportation}{%
\subsection{Gwinnett County and
Transportation}\label{gwinnett-county-and-transportation}}

Gwinnett County sits in the north-central part of Georgia in the United
States (Fast Facts \textbar{} Gwinnett County, n.d.). The estimated
population was 936,250 in 2019 (U.S. Census Bureau QuickFacts, n.d.),
with approximately 55\% identified as white, 29.3\% identified as Black
or African American, 12.4\% as Asian, and 0.9\% identified as other
races. For ethnicity, 21.5\% identified themselves as Hispanic or
Latino.

The primary transportation means for Gwinnett County residents is
driving. According to Census Bureau statistics in 2018 (Census - Means
of Transportation to Work by Vehicles Available, n.d.), of the 456,465
households who are workers over 16 years old, over 90\% drive alone or
carpool to work. On the other hand, only 0.8\% take public transit to
work. Since approximately 4.9\% of Americans use public transportation
to commute (Principal Means of Transportation to Work \textbar{} Bureau
of Transportation Statistics, n.d.), the usage of public transit in
Gwinnett is lower than other parts of the country.

The major roads in Gwinnett include two interstate highways, three U.S.
routes, and eighteen state routes (``Gwinnett County, Georgia,'' 2020).
In the public transit sector, Gwinnett County Transit currently operates
seven Local bus routes and five Express routes (Routes and Schedules
\textbar{} Gwinnett County, n.d.). The heavy rail transit (HRT) operated
by the Metropolitan Atlanta Rapid Transit Authority (MARTA) does not go
through Gwinnett. Yet, the nearest rail station is very close to the
border, and several bus lines start from there (Connect Gwinnett Transit
Plan Recommendations Report, 2018).

\hypertarget{connect-gwinnett-transit-plan}{%
\subsection{Connect Gwinnett: Transit
Plan}\label{connect-gwinnett-transit-plan}}

The population in Gwinnett has nearly doubled from 2000 to 2016, but
public transit service has not expanded (Connect Gwinnett Transit Plan
Recommendations Report, 2018). As a result, Gwinnett County Transit
drafted a public transit expansion plan named ``Connect Gwinnett:
Transit Plan'' in 2018. All of the following information comes from its
recommendations report (2018).

The plan consists of four stages: short-range (2019 - 2024), mid-range
(2025 -- 2029), long-range phase 1 (2030 -- 2049), and long-range phase
2 (2050 onward). The most exciting improvement is the first HRT service
proposed to build during long-range phase 1. It is an extension of an
existing line from MARTA.

Nevertheless, I decided to study the short-range plan in this paper in
order to minimize the effect of discounting on future costs and
benefits. Short-range recommendations target on the most immediate
demands. During this period, Gwinnet transit service will increase to
seventeen local bus routes, nine express commuter bus routes, two direct
connect lines, and two flex routes. Direct connect is a new service
introduced in the short-range that takes passengers to and from MARTA
stations with limited stops to shorten travel time. Flex is an on-demand
service that carries passengers upon request.

The entire Connect Gwinnett: Transit Plan is estimated to cost 527.7
million dollars. The plan is going to be funded mainly by sales tax
revenue, assuming a 1.5\% increase of tax in the first five years and a
1\% increase afterward.

\hypertarget{referendum}{%
\subsection{Referendum}\label{referendum}}

A special referendum was held on March 19th, 2019 to decide whether the
county will join MARTA and expand the public transit system following
Connect Gwinnett: Transit Plan. A vote for yes ``would be a vote in
support of ratifying Gwinnett's pending transit service contract with
MARTA --- and one in favor of enacting a new 1 percent sales tax to pay
for transit projects and operations in the county.'' (Estep, 2019)

The referendum eventually failed. Among the 92,243 votes, 42,156 voted
yes, and 50,087 voted no (Results - Gwinnett - Election Night Reporting,
2019). The turnout rate was 16.98\%.

Some argue that the failure could be due to the delay of the referendum
from November 2018 to March 2019 (Yeomans, 2018). The change from voting
in a general election to a special referendum would reduce voter turnout
and affect the result.

\hypertarget{data-methods}{%
\section{Data \& Methods}\label{data-methods}}

\hypertarget{conceptual-model}{%
\subsection{Conceptual model}\label{conceptual-model}}

This paper sets out to find the factors that influence voting results in
this referendum. According to previous research, sociodemographic
elements can influence people's voting decisions in the referendum. For
example, the effect of income is mixed: on the one hand, people with
higher income will pay a smaller portion of their earnings for the
implementation of the plan; on the other hand, they will pay a larger
amount of tax. Bollino (2008) finds a positive correlation between
income and people's willingness to pay for renewable resources.
Burkhardt and Chan (2017) separate the influence of income from tax, and
discover their opposite effects on voting. Therefore, it is worth
considering the relationship between income and percentage of supporters
in this referendum. Voters' partisanship attachment is found to be a
significant factor as well in Burkhardt and Chan's (2017) paper. Areas
with higher proportions of Republicans are less supportive of fiscally
costly propositions. In my project, it can be hypothesized that tracts
that have a higher proportion of Trump supporters tend to have a lower
percentage of agreement to the proposal.

In addition, some factors related to transportation can intuitively
shape people's attitudes towards public transit. For example, the areas
in which people do not use public transit at all might have a higher
percentage of refusal of the proposal, because they have alternative
transportation means. People who have to travel a long time to work are
more likely to support the extension plan if it helps save time. These
two factors serve as controls in the models.

Finally, people favor the proposition if it benefits them. Specifically,
tracts that are not covered by public transport at present but will be
covered in the expansion plan are predicted to support the proposal
more.

\hypertarget{data}{%
\subsection{Data}\label{data}}

The ballot results of the Gwinnett County referendum on Mar 19th, 2019
is obtained from a website powered by Scytl, a trusted source of
election outcomes. The cross-sectional dataset contains the voting
information of all 157 precincts in the county. The dependent variable
-- the proportion of supporters of the referendum, and the voter turnout
rate are calculated from this data source.

The result of the 2016 Presidential Election is chosen to reflect
partisanship. A cross-sectional precinct-level election data is obtained
from the MIT Election Data and Science Lab website. The number of votes
for Trump at each precinct in Gwinnett county comes from this dataset.

Next, cross-sectional sociodemographic characteristics at the census
tract level are found in Census Bureau via an R package called
\texttt{tidycensus}. The data comes from the American Community Survey
5-year estimate published in 2018. The median age and median income are
taken from here. In addition, the proportion of the population that is
white, the percentage of people who go to work by public transportation,
and the percentage of people who travel more than an hour to work are
calculated by dividing the relevant variables by the total population or
the survey sample size.

The information on whether the tract enjoys the proximity of public
transportation now and future can be acquired from spatial maps and
analyses. First, a precinct-level shapefile of Gwinnett County made in
2018 is obtained from the Georgia General Assembly. Gwinnett County maps
with current and proposed future public transit systems are available on
the Gwinnett County government website. I select the short-range
expansion plan (Y2019 -- 2024) because the cost and benefit of the
expansion in the far future are discounted more. PDF maps are
georeferenced in QGIS software, and then exported as spatial data
readable in R. Five-hundred-meter buffer zones are created around bus
and railway routes. The categorical variable \emph{current\_plan} has
four levels: 1 represents no transportation near this tract at present
and in the future; 2 represents the tracts that are accessible to public
transit currently but not in the future; 3 stands for the tracts that do
not have transit at present but will do in the expansion plan; 4
represents the tracts that have and will have public transit for now and
for the future.

As noted above, both referendum and 2016 election data are collected at
the precinct level. However, the other datasets are performed at the
census tract level. Therefore, referendum and election data are
redistributed by the areas shared by the precinct and the tract,
following Burkhardt and Chan (2017). See the data appendix for detailed
steps of transformation.

The final dataset joins the datasets above by census tract. It is
cross-sectional, measured with the unit of the census tract. Since there
are 113 census tracts in Gwinnett, the dataset has 113 observations,
with no missing data. A description of the variables can be found in
\emph{Table 1}.

\begin{table}
\centering
\caption{Variable definitions}
\label{variableDefinitions}
\begin{tabular}{ll}
\hline
\hline
Variable name      & Description                                   \\
\hline
GEOID          & The geographic identifier of the census tract                 \\
medage        & The median age of the population in the tract               \\
medincome        & The median income of the population in the tract               \\
white\_pct           & The percentage of white population in the tract      \\
public\_pct          & The percentage of people who go to work by public transportation (excluding taxi or cab)      \\
time\_pct           & The percentage of people who travel more than an hour to work      \\
trump\_pct          & The estimated percentage of votes for Donald Trump in that tract      \\
voter\_turnout  & The estimated percentage of voters who voted in this referendum in the tract \\
yes\_pct           & The estimated percentage of voters who voted yes in this referendum in the tract      \\
plan\_yes          & Whether the tract is covered by the public transportation now and in the short-range, \\ & defined by whether any transportation is available within 500 meters. \\ & 1 stands for the tract doesn't have transit both now and in the short-range plan. \\ & 2 stands for the tract has transit now but not in the short-range plan. \\ & 3 stands for the tract that doesn't have transit now and will have in the future.\\ & 4 stands for the tract that has public transit both now and in the future      \\

\hline
\end{tabular}
\end{table}

\begin{table}[!htbp] \centering 
  \caption{Summary statistics} 
  \label{summaryStats} 
\begin{tabular}{@{\extracolsep{5pt}}lccccccc} 
\\[-1.8ex]\hline 
\hline \\[-1.8ex] 
Statistic & \multicolumn{1}{c}{N} & \multicolumn{1}{c}{Mean} & \multicolumn{1}{c}{St. Dev.} & \multicolumn{1}{c}{Min} & \multicolumn{1}{c}{Pctl(25)} & \multicolumn{1}{c}{Pctl(75)} & \multicolumn{1}{c}{Max} \\ 
\hline \\[-1.8ex] 
medage & 113 & 35.56 & 4.58 & 26 & 32.8 & 38.8 & 52 \\ 
medincome & 113 & 69,439.24 & 24,358.44 & 33,020 & 51,429 & 82,845 & 156,136 \\ 
white\_pct & 113 & 0.48 & 0.15 & 0.17 & 0.38 & 0.61 & 0.89 \\ 
public\_pct & 113 & 0.01 & 0.01 & 0 & 0.002 & 0.02 & 0.06 \\ 
time\_pct & 113 & 0.16 & 0.05 & 0.04 & 0.12 & 0.20 & 0.31 \\ 
trump\_pct & 113 & 0.40 & 0.15 & 0.11 & 0.27 & 0.52 & 0.69 \\ 
voter\_turnout & 113 & 0.16 & 0.06 & 0.05 & 0.13 & 0.18 & 0.37 \\ 
yes\_pct & 113 & 0.53 & 0.14 & 0.27 & 0.42 & 0.61 & 0.84 \\ 
\hline \\[-1.8ex] 
\end{tabular} 
\end{table}

\includegraphics{Zhong_paper_files/figure-latex/create correlation matrix-1.pdf}

\emph{Table 2} lists the summary statistics. Median income is skewed to
the right, with a few tracts demonstrating high income levels. Such a
pattern of inequality is universally observed.

The usage of public transit is low. On average, only one percent of the
population relies on public transportation to go to work. As proved by
later analyses, the distribution is highly skewed, and will be
transformed in some models.

The variable \emph{current\_plan} is not in \emph{Table 2} because it is
a categorical variable. In sum, 31 tracts do not have access to public
transport within 500 meters, both now and in the short-range. There is
one tract that is categorized as 2, indicating that it has public
transit at present, but not in the short-range proposition. It could be
due to the plan of reducing circuitous routing. Twenty-two tracts do not
enjoy the proximity of public transport now, but will in the
short-range. Lastly, 59 tracts have public transit available both at
present and in the short-range plan. Since a level of this variable
contains only one value (``2''), problems with degrees of freedom will
potentially arise.

A correlation matrix is created from \texttt{corrplot} package to
investigate the correlation between each pair of factors (see
\emph{Figure 1}). Positive correlations are in blue colors, and negative
correlations are in red colors. The magnitude is reflected by the color
intensity and the size of the dot. Non-significant (p \textgreater{}
0.05) correlations are marked with a cross. \emph{Current\_plan} is
omitted in the matrix because it is categorical. The dependent variable
\emph{yes\_pct} is significantly correlated with every independent
variable. \emph{Medage}, \emph{voter\_turnout}, \emph{medincome},
\emph{white\_pct}, and \emph{trump\_pct} are all strongly positively
correlated with each other, and all of them are negatively associated
with \emph{yes\_pct}. Given the number of strong, significant
correlations among the factors, it is essential to check collinearity in
the regression model.

The major limitation of the data is the unit conversion from precinct to
census tract. Such a method assumes that residents in one precinct have
the same characteristics, and population density is identical. Clearly,
the assumptions cannot be satisfied in a real dataset.

\hypertarget{model-specification}{%
\subsection{Model specification}\label{model-specification}}

Two models are tested in the analysis.

Model 1 adopts linear regression model with all the original variables:

\(yes\_pct_t = \beta_0+\beta_1medincome_t+\beta_2white\_pct_t+\beta_3trump\_pct_t+\beta_4current\_plan_t+\beta_5medage_t+\beta_6public\_pct_t+\beta_7time\_pct_t+\beta_8voter\_turnout_t+\epsilon_t\)

where \(t\) indexes census tract. Since all variables are measured in
the same scope, no fixed effects are tested. The dependent variable is
\emph{yes\_pct}, the proportion of supporters of the referendum in a
tract. Independent variables include \emph{medincome},
\emph{white\_pct}, \emph{trump\_pct}, and \emph{current\_plan}. For more
information about these variables, refer to \emph{Table 1}. I
hypothesize that the effect of \emph{medincome} is ambiguous. As
explained in the Conceptual Model section, people with higher income pay
a larger amount of tax, but the tax takes up a smaller portion of their
earnings compared to those with lower income. \emph{White\_pct} is
hypothesized to have a negative effect on \emph{yes\_pct}, based on the
article stating that the white population had historically rejected a
similar proposition twice (Binkovitz, 2017). A higher percentage of
Trump supporters is expected to predict a lower percentage of votes for
``yes'' in the referendum, as evidenced by common sense and previous
literature (Kinsey et al., 2010). Finally, since tracts that do not have
public transit now and will have it in the short-range benefit the most
from the proposition, I predict that the tracts with this feature will
have a higher level of \emph{yes\_pct} than the others.

\emph{Medage}, \emph{public\_pct}, \emph{time\_pct}, and
\emph{voter\_turnout} add to the model as controls. For example, tracts
with higher voter turnout are hypothesized to have a lower
\emph{yes\_pct}, according to the rule of loss aversion -- people who
believe the referendum incur losses to them are inclined to participate
in the referendum actively and vote against it, but people who like the
proposition are less motivated to vote in nature.

\emph{Medincome}, \emph{public\_pct} and \emph{voter\_turnout} are found
to be positively skewed. Thus, Model 2 uses multiple linear regression
after log or square root transformation on these three variables. Among
them, since many of the values in \emph{public\_pct} are 0, a constant
has to add to the variable before taking log transformation.

\(yes\_pct_t = \beta_0+\beta_1log(medincome_t)+\beta_2white\_pct_t+\beta_3trump\_pct_t+\beta_4current\_plan_t+\beta_5medage_t+\beta_6log(public\_pct_t + 0.01)+\beta_7time\_pct_t+\beta_8\sqrt{voter\_turnout_t}+\epsilon_t\)

\hypertarget{results-discussion}{%
\section{Results \& Discussion}\label{results-discussion}}

\hypertarget{regression-results}{%
\subsection{Regression results}\label{regression-results}}

\begin{table}[!htbp] \centering 
  \caption{Primary regression} 
  \label{Primary} 
\begin{tabular}{@{\extracolsep{5pt}}lc} 
\\[-1.8ex]\hline 
\hline \\[-1.8ex] 
 & \multicolumn{1}{c}{\textit{Dependent variable:}} \\ 
\cline{2-2} 
\\[-1.8ex] & yes\_pct \\ 
\hline \\[-1.8ex] 
 medincome & 0.00000 \\ 
  & (0.00000) \\ 
  & \\ 
 white\_pct & 0.114$^{**}$ \\ 
  & (0.053) \\ 
  & \\ 
 trump\_pct & $-$0.886$^{***}$ \\ 
  & (0.057) \\ 
  & \\ 
 current\_plan2 & 0.019 \\ 
  & (0.044) \\ 
  & \\ 
 current\_plan3 & $-$0.032$^{***}$ \\ 
  & (0.012) \\ 
  & \\ 
 current\_plan4 & 0.016 \\ 
  & (0.012) \\ 
  & \\ 
 medage & 0.003$^{*}$ \\ 
  & (0.001) \\ 
  & \\ 
 public\_pct & 0.817$^{**}$ \\ 
  & (0.338) \\ 
  & \\ 
 time\_pct & $-$0.306$^{***}$ \\ 
  & (0.090) \\ 
  & \\ 
 voter\_turnout & $-$0.326$^{**}$ \\ 
  & (0.126) \\ 
  & \\ 
 Constant & 0.804$^{***}$ \\ 
  & (0.047) \\ 
  & \\ 
\hline \\[-1.8ex] 
Observations & 113 \\ 
R$^{2}$ & 0.917 \\ 
Adjusted R$^{2}$ & 0.909 \\ 
Residual Std. Error & 0.042 (df = 102) \\ 
F Statistic & 113.224$^{***}$ (df = 10; 102) \\ 
\hline 
\hline \\[-1.8ex] 
\textit{Note:}  & \multicolumn{1}{l}{$^{*}$p$<$0.1; $^{**}$p$<$0.05; $^{***}$p$<$0.01} \\ 
\end{tabular} 
\end{table}

\emph{Table 3} provides the results of Model 1. The equation is
statistically significant (F(10, 102) = 113.22, p \textless{} 0.05).
Adjusted \(R^2\) is 0.909, indicating that the independent variables in
this specification explain a large portion of the variance in the
dependent variable. Consistent with the hypothesis, \emph{medincome} is
not a significant predictor of \emph{yes\_pct}. \emph{Trump\_pct}
significantly predicts \emph{yes\_pct} as expected, and results in a
large coefficient: a one percent increase in the proportion of Trump
supporters decreases the percentage of voters in favor of the referendum
by 0.886\%, holding other variables constant.

However, unlike the beliefs in previous articles, \emph{white\_pct}
significantly predicts \emph{yes\_pct} in a positive direction. A higher
percentage of the white population in the tract is associated with
higher support of the proposition when holding other explanatory
variables constant.

Another unanticipated result is \emph{current\_plan}. In contrast to
baseline tracts that do not access public transit service both now and
in the short-range (\emph{current\_plan 1}), only \emph{current\_plan 3}
tracts differ from them significantly, in the opposite direction to the
hypothesis. That is to say, tracts planned to be newly added to the
public transit service have a significantly higher rejection of the
proposition than the other tracts. The control variables are all
significant factors. \emph{Medage} and \emph{public\_pct} positively
predict the level of \emph{yes\_pct}, while \emph{time\_pct} and
\emph{voter\_turnout} negatively predict \emph{yes\_pct}.

\begin{table}[!htbp] \centering 
  \caption{Transformed regression} 
  \label{Transformed} 
\begin{tabular}{@{\extracolsep{5pt}}lc} 
\\[-1.8ex]\hline 
\hline \\[-1.8ex] 
 & \multicolumn{1}{c}{\textit{Dependent variable:}} \\ 
\cline{2-2} 
\\[-1.8ex] & yes\_pct \\ 
\hline \\[-1.8ex] 
 log\_medincome & $-$0.008 \\ 
  & (0.022) \\ 
  & \\ 
 white\_pct & 0.117$^{**}$ \\ 
  & (0.051) \\ 
  & \\ 
 trump\_pct & $-$0.857$^{***}$ \\ 
  & (0.059) \\ 
  & \\ 
 current\_plan2 & 0.021 \\ 
  & (0.043) \\ 
  & \\ 
 current\_plan3 & $-$0.033$^{***}$ \\ 
  & (0.012) \\ 
  & \\ 
 current\_plan4 & 0.014 \\ 
  & (0.011) \\ 
  & \\ 
 medage & 0.004$^{**}$ \\ 
  & (0.002) \\ 
  & \\ 
 log\_public\_pct & 0.022$^{***}$ \\ 
  & (0.008) \\ 
  & \\ 
 time\_pct & $-$0.281$^{***}$ \\ 
  & (0.090) \\ 
  & \\ 
 sqrt\_voter\_turnout & $-$0.319$^{***}$ \\ 
  & (0.103) \\ 
  & \\ 
 Constant & 1.032$^{***}$ \\ 
  & (0.219) \\ 
  & \\ 
\hline \\[-1.8ex] 
Observations & 113 \\ 
R$^{2}$ & 0.920 \\ 
Adjusted R$^{2}$ & 0.912 \\ 
Residual Std. Error & 0.041 (df = 102) \\ 
F Statistic & 117.561$^{***}$ (df = 10; 102) \\ 
\hline 
\hline \\[-1.8ex] 
\textit{Note:}  & \multicolumn{1}{l}{$^{*}$p$<$0.1; $^{**}$p$<$0.05; $^{***}$p$<$0.01} \\ 
\end{tabular} 
\end{table}

Model 2 uses multiple linear regression as well, except that the skewed
variables \emph{medincome}, \emph{public\_pct}, and
\emph{voter\_turnout} are transformed. The overall outcome is identical
to Model 1 (see \emph{Table 4}). After log transformation, the effect of
median income is still insignificant.

A series of assumptions are examined. First, linearity and
homoscedasticity assumptions are met, as illustrated by the residual
plots. Next, because several factors correlate with each other (see
\emph{Figure 1}), VIFs are calculated to detect multicollinearity. Since
all of the independent variables exhibit a VIF below 5 in both models,
no collinearity issue is detected. Lastly, there is a potential fault of
using linear regression in this dataset. The value of the dependent
variable \emph{yes\_pct} is restricted to the range of 0 to 1, because
it represents a percentage. On the other hand, the predicted outcome
from linear regression is unbounded. As a result, I calculate the
predicted values from the two models with the actual datasets. The
predicted outcome is close to the actual value of \emph{yes\_pct}, all
between 0 and 1. Then, I continue testing by finding the minimum and
maximum values of each independent variable in the dataset (to see the
actual minimum and maximum values, see \emph{Table 2}), and plug them
into the models accordingly to gain the extreme predicted
\emph{yes\_pct}. The extreme values for Model 1 are 0.09 and 0.96, and
0.11 and 0.95 in Model 2, all within the interval {[}0,1{]}. Therefore,
the range of the dependent variable in these two regression models meets
the assumption.

\hypertarget{discussion}{%
\subsection{Discussion}\label{discussion}}

Overall, most of the explanatory and control variables predict the
proportion of supporters of the expansion proposition as hypothesized.

In both models, the median income does not significantly predict
\emph{yes\_pct}. The finding is consistent with the hypothesis that the
effect of income level is complicated by taxes. If more information on
the financing plan of the expansion and the details of taxation were
available, it would be possible to separate the influence of taxation
from income.

Partisanship attachment is a significant factor, even after controlling
for other variables such as income and race. The tracts that have a
higher proportion of votes for Trump during the 2016 Election tend to
reject the expansion of public transport more. Republicans are usually
reluctant to agree on public spending due to the increase in tax
incurred.

As for the control variables, the median age positively associates with
\emph{yes\_pct}, but the coefficient is relatively small. The percentage
of people taking public transit to work predicts \emph{yes\_pct}
positively as well. People who do not regularly utilize public transit
benefit less from the expansion because they are likely to have other
transportation available, and switching to public transportation
generates costs such as registering, and spending time to get familiar
with the locations and operation. \emph{Time\_pct} negatively predicts
\emph{yes\_pct}, indicating that the areas in which more people spend
over an hour to work tend to disapprove with the proposal. Voter turnout
rate is negatively related to the dependent variable, conforming with
previous research conclusion that those who disagree with the proposal
are more likely to express their opinions.

\hypertarget{complications-with-the-current_plan-variable}{%
\subsubsection{\texorpdfstring{Complications with the
\emph{current\_plan}
variable}{Complications with the current\_plan variable}}\label{complications-with-the-current_plan-variable}}

\includegraphics{Zhong_paper_files/figure-latex/create maps-1.pdf}

One of the most unforeseen findings is that the tracts that are newly
included in the public transit system in the short-range plan have a
significantly higher rate of refusal to the proposition. Intuitively,
they should support the plan the most, because they are the direct
beneficiaries.

A possible explanation is they are worried about the short-term
disturbances due to the construction such as traffic congestion and
noises. However, since short-range expansion proposal does not include
services that require heavy construction, this speculation is less
persuasive.

Another supposition is that residents in these tracts do not like the
idea of ``Connecting Gwinnett.'' As can be seen from \emph{Figure 2},
the tracts marked with purple in the first map are mostly high in
\emph{trump\_pct}, \emph{white\_pct}, and low in \emph{yes\_pct}. It is
possible that some sort of segregation still exists, that people in
these areas refuse to get connected with other communities that are less
``white,'' and such discrimination is not entirely captured by the
existing explanatory variables. In future research, I recommend adding
another factor such as the number of hate crimes against racial or
ethnic minorities, as a measure of ``the level of racism.''

Previous rejections of MARTA in Gwinnett can support this speculation.
Since 1971, Gwinnett residents had refused to develop public transit
with MARTA because they preferred driving over traveling by public
transit with other Black people (Binkovitz, 2017). They were afraid that
their white neighborhoods would be disturbed by racial minorities via
public transportation.

\hypertarget{race}{%
\subsubsection{Race}\label{race}}

The effect of the proportion of the white population is unexpected. The
model suggests that holding other variables constant, a higher
\emph{white\_pct} would predict more support for the referendum,
contradicting with the previous articles (Binkovitz, 2017; Karner,
2019). I believe it is because the influence of racial discrimination is
already explained in the variables such as \emph{trump\_pct}. Therefore,
when holding \emph{trump\_pct} and other variables constant, the effect
of racism is controlled, and thus \emph{white\_pct} would predict in a
different direction than anticipated.

The reason for \emph{white\_pct} as a positive predictor might trace
back to the beginning of public transportation service in the
metropolitan Atlanta area. In the mid-1900s, white households were
encouraged to move from urban to suburban areas from transportation and
housing subsidies. As a result, Atlanta city was populated by Black
families (Binkovitz, 2017). To fulfill the white's demands to connect
with the city, MARTA was created. The bias towards the white population
in the first place could potentially explain why the white favor the
proposition after partisanship factor is controlled.

\hypertarget{limitation-future-direction}{%
\subsubsection{Limitation \& Future
direction}\label{limitation-future-direction}}

This research has several limitations. First of all, I have only
examined the short-range plan. As I have discussed in Section 3, one of
the most critical propositions is the construction of HRT in the
long-range. Future research could look at other plans to test if they
have a different influence on voters' decisions.

Moreover, census data is collected from every household, but not
everyone participated in the referendum. The demographic distribution of
the entire tract can be different from that of the voters in the tract.
If some groups disproportionally participate more or less actively than
the others, the data from Census Bureau cannot accurately describe the
characteristics of the voters.

In order to improve the model, more variables can add to the regression.
For example, the effect of race can be further specified. This paper
only divides race into white and non-white. Future research could lay
out the minorities as variables. For example, Asian or Asian Americans
might perceive public transit in a different way from Black or African
Americans. Other variables, such as education level, could also
significantly predict voting outcomes and are worth testing in the
future.

Another research direction is to study the determinants of voter turnout
rate. It would be interesting to see if the tracts with more Trump
supporters also demonstrate higher voter turnout.

\hypertarget{conclusion}{%
\section{Conclusion}\label{conclusion}}

In sum, in a public transit referendum, a number of sociodemographic
characteristics can predict voting behavior. Firstly, partisanship is a
significant predictor. Republicans, characterized as Trump supporters
during the 2016 Election, have a higher tendency to oppose the
referendum. The tracts with a higher percentage of the white population
are predicted to support the referendum more, when other variables are
controlled. This could reflect a racial bias towards the white
population in the history of public transit services in the metropolitan
Atlanta area. The tracts that are newly added to the public transport
system counterintuitively reject the referendum more. It suggests that
the local white residents would refuse to connect with other parts of
the area that are less white.

A new ballot on the public transit expansion is possibly scheduled to
hold in November 2020. Since the voting is going to happen in the
general election, it is expected that there will be a higher turnout
rate and more diversity in voters. The findings in this paper explicitly
suggest that Republicans and people living in tracts newly included in
the public transit service oppose the referendum. Campaigns for the
public transit expansion plan could target this population. For example,
campaigners can persuade the residents in the tracts that will be added
to public transit to support by advocating the economic and
environmental benefits of utilizing public transit.

\hypertarget{references}{%
\section{References}\label{references}}

Allen, S., Gaunt, M., \& Rye, T. (2006). An investigation into the
reasons for the rejection of congestion charging by the citizens of
Edinburgh. \emph{Trasporti Europei, 32}, 95--113.

\emph{Ambient (outdoor) air pollution.} (n.d.). Retrieved May 3, 2020,
from
\url{https://www.who.int/news-room/fact-sheets/detail/ambient-(outdoor)-air-quality-and-health}

Binkovitz, L. (2017, February 8). \emph{New Study Examines How Historic
Racism Shaped Atlanta's Transportation Network.} The Kinder Institute
for Urban Research.
\url{https://kinder.rice.edu/2017/02/08/new-study-examines-how-historic-racism-shaped-atlantas-transportation-network}

Bollino, C. A. (2009). The Willingness to Pay for Renewable Energy
Sources: The Case of Italy with Socio-demographic Determinants.* Energy
Journal,* 30(2), 81--96.

Burkhardt, J., \& Chan, N. W. (2017). The Dollars and Sense of Ballot
Propositions: Estimating Willingness to Pay for Public Goods Using
Aggregate Voting Data. \emph{Journal of the Association of Environmental
and Resource Economists, 4}(2), 479--503.

\emph{Census---Means of Transportation to Work by Vehicles Available}.
(n.d.). Retrieved May 7, 2020, from
\url{https://data.census.gov/cedsci/table?q=transportation\%20means\%20to\%20work\&g=0500000US13135\&hidePreview=true\&tid=ACSDT1Y2018.B08141\&t=Transportation\&vintage=2018}

Chen, Y., \& Whalley, A. (2012). Green Infrastructure: The Effects of
Urban Rail Transit on Air Quality. \emph{American Economic Journal:
Economic Policy, 4}(1), 58--97. \url{https://doi.org/10.1257/pol.4.1.58}

\emph{Connect Gwinnett Transit Plan Recommendations Report}. (2018).

Estep, T. (2019, March 15). \emph{Gwinnett's MARTA referendum: A
comprehensive voter's guide}. Ajc.
\url{https://www.ajc.com/news/local-govt--politics/gwinnett-upcoming-marta-referendum-comprehensive-voter-guide/CVj5YhwsvzesGoX29o0xvL/}

\emph{Fast Facts \textbar{} Gwinnett County.} (n.d.). Retrieved May 7,
2020, from
\url{https://www.gwinnettcounty.com/web/gwinnett/aboutgwinnett/fastfacts}

Gwinnett County, Georgia. (2020). In \emph{Wikipedia}.
\url{https://en.wikipedia.org/w/index.php?title=Gwinnett_County,_Georgia\&oldid=954101825}

Karner, A. (2019, April 23). MARTA Rebranding Rooted in a Racialized
History of Public Transit. \emph{Atlanta Studies}.
\url{https://www.atlantastudies.org/2019/04/23/alex-karner-marta-rebranding-rooted-in-a-racialized-history-of-public-transit/}

Kinsey, B. S., Bartling, H., Peterson, A. F., \& Baybeck, B. P. (2010).
Location of Public Goods and the Calculus of Voting: The Seattle
Monorail Referendum. \emph{Social Science Quarterly, 91}(3), 741--761.
\url{https://doi.org/10.1111/j.1540-6237.2010.00717.x}

Manville, M., \& Cummins, B. (2015). Why do voters support public
transportation? Public choices and private behavior.
\emph{Transportation, 42}(2), 303--332.
\url{https://doi.org/10.1007/s11116-014-9545-2}

Manville, M., \& Levine, A. S. (2018). What motivates public support for
public transit? \emph{Transportation Research Part A: Policy and
Practice, 118,} 567--580.
\url{https://doi.org/10.1016/j.tra.2018.10.001}

\emph{Principal Means of Transportation to Work \textbar{} Bureau of
Transportation Statistics. }(n.d.). Retrieved May 7, 2020, from
\url{https://www.bts.gov/content/principal-means-transportation-work}

\emph{Results---Gwinnett---Election Night Reporting.} (2019).
\url{https://results.enr.clarityelections.com/GA/Gwinnett/94961/Web02.225391/\#/}

\emph{Routes and Schedules \textbar{} Gwinnett County. } (n.d.).
Retrieved May 7, 2020, from
\url{https://www.gwinnettcounty.com/web/gwinnett/Departments/Transportation/GwinnettCountyTransit/RoutesandSchedules}

\emph{Taking Action. } (n.d.). MARTA for Gwinnett. Retrieved May 8,
2020, from \url{https://marta4gwinnett.org/takingaction}

\emph{U.S. Census Bureau QuickFacts: Gwinnett County, Georgia. } (n.d.).
Retrieved May 7, 2020, from
\url{https://www.census.gov/quickfacts/fact/table/gwinnettcountygeorgia/IPE120218\#qf-headnote-a}

Ward, B. G. (2009). Disaggregating Race and Ethnicity: Toward a Better
Understanding of the Social Impacts of Transport Decisions. \emph{Public
Works Management \& Policy, 13} (4), 354--360.
\url{https://doi.org/10.1177/1087724X09334495}

\emph{WHO \textbar{} Air pollution. } (n.d.). WHO; World Health
Organization. Retrieved May 3, 2020, from
\url{http://www.who.int/sustainable-development/transport/health-risks/air-pollution/en/}

Yeomans, C. (2018, August 2). \emph{MARTA referendum in March---Rather
than November---Could cost Gwinnett \$500,000. } Gwinnett Daily Post.
\url{https://www.gwinnettdailypost.com/local/marta-referendum-in-march---rather-than-november---could-cost-gwinnett-500/article_f6c8b02e-874e-54b4-9c42-a20278e19238.html}

Zheng, S., Zhang, X., Sun, W., \& Wang, J. (2019). The effect of a new
subway line on local air quality: A case study in Changsha.
\emph{Transportation Research Part D: Transport and Environment, 68,
}26--38. \url{https://doi.org/10.1016/j.trd.2017.10.004}

\end{document}
