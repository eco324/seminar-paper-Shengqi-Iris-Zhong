% Options for packages loaded elsewhere
\PassOptionsToPackage{unicode}{hyperref}
\PassOptionsToPackage{hyphens}{url}
%
\documentclass[
]{article}
\usepackage{lmodern}
\usepackage{amssymb,amsmath}
\usepackage{ifxetex,ifluatex}
\ifnum 0\ifxetex 1\fi\ifluatex 1\fi=0 % if pdftex
  \usepackage[T1]{fontenc}
  \usepackage[utf8]{inputenc}
  \usepackage{textcomp} % provide euro and other symbols
\else % if luatex or xetex
  \usepackage{unicode-math}
  \defaultfontfeatures{Scale=MatchLowercase}
  \defaultfontfeatures[\rmfamily]{Ligatures=TeX,Scale=1}
\fi
% Use upquote if available, for straight quotes in verbatim environments
\IfFileExists{upquote.sty}{\usepackage{upquote}}{}
\IfFileExists{microtype.sty}{% use microtype if available
  \usepackage[]{microtype}
  \UseMicrotypeSet[protrusion]{basicmath} % disable protrusion for tt fonts
}{}
\makeatletter
\@ifundefined{KOMAClassName}{% if non-KOMA class
  \IfFileExists{parskip.sty}{%
    \usepackage{parskip}
  }{% else
    \setlength{\parindent}{0pt}
    \setlength{\parskip}{6pt plus 2pt minus 1pt}}
}{% if KOMA class
  \KOMAoptions{parskip=half}}
\makeatother
\usepackage{xcolor}
\IfFileExists{xurl.sty}{\usepackage{xurl}}{} % add URL line breaks if available
\IfFileExists{bookmark.sty}{\usepackage{bookmark}}{\usepackage{hyperref}}
\hypersetup{
  pdftitle={The Influence of Individual Characterisitcs on Public Transportation Planning},
  pdfauthor={Iris Zhong},
  hidelinks,
  pdfcreator={LaTeX via pandoc}}
\urlstyle{same} % disable monospaced font for URLs
\usepackage[margin=1in]{geometry}
\usepackage{color}
\usepackage{fancyvrb}
\newcommand{\VerbBar}{|}
\newcommand{\VERB}{\Verb[commandchars=\\\{\}]}
\DefineVerbatimEnvironment{Highlighting}{Verbatim}{commandchars=\\\{\}}
% Add ',fontsize=\small' for more characters per line
\usepackage{framed}
\definecolor{shadecolor}{RGB}{248,248,248}
\newenvironment{Shaded}{\begin{snugshade}}{\end{snugshade}}
\newcommand{\AlertTok}[1]{\textcolor[rgb]{0.94,0.16,0.16}{#1}}
\newcommand{\AnnotationTok}[1]{\textcolor[rgb]{0.56,0.35,0.01}{\textbf{\textit{#1}}}}
\newcommand{\AttributeTok}[1]{\textcolor[rgb]{0.77,0.63,0.00}{#1}}
\newcommand{\BaseNTok}[1]{\textcolor[rgb]{0.00,0.00,0.81}{#1}}
\newcommand{\BuiltInTok}[1]{#1}
\newcommand{\CharTok}[1]{\textcolor[rgb]{0.31,0.60,0.02}{#1}}
\newcommand{\CommentTok}[1]{\textcolor[rgb]{0.56,0.35,0.01}{\textit{#1}}}
\newcommand{\CommentVarTok}[1]{\textcolor[rgb]{0.56,0.35,0.01}{\textbf{\textit{#1}}}}
\newcommand{\ConstantTok}[1]{\textcolor[rgb]{0.00,0.00,0.00}{#1}}
\newcommand{\ControlFlowTok}[1]{\textcolor[rgb]{0.13,0.29,0.53}{\textbf{#1}}}
\newcommand{\DataTypeTok}[1]{\textcolor[rgb]{0.13,0.29,0.53}{#1}}
\newcommand{\DecValTok}[1]{\textcolor[rgb]{0.00,0.00,0.81}{#1}}
\newcommand{\DocumentationTok}[1]{\textcolor[rgb]{0.56,0.35,0.01}{\textbf{\textit{#1}}}}
\newcommand{\ErrorTok}[1]{\textcolor[rgb]{0.64,0.00,0.00}{\textbf{#1}}}
\newcommand{\ExtensionTok}[1]{#1}
\newcommand{\FloatTok}[1]{\textcolor[rgb]{0.00,0.00,0.81}{#1}}
\newcommand{\FunctionTok}[1]{\textcolor[rgb]{0.00,0.00,0.00}{#1}}
\newcommand{\ImportTok}[1]{#1}
\newcommand{\InformationTok}[1]{\textcolor[rgb]{0.56,0.35,0.01}{\textbf{\textit{#1}}}}
\newcommand{\KeywordTok}[1]{\textcolor[rgb]{0.13,0.29,0.53}{\textbf{#1}}}
\newcommand{\NormalTok}[1]{#1}
\newcommand{\OperatorTok}[1]{\textcolor[rgb]{0.81,0.36,0.00}{\textbf{#1}}}
\newcommand{\OtherTok}[1]{\textcolor[rgb]{0.56,0.35,0.01}{#1}}
\newcommand{\PreprocessorTok}[1]{\textcolor[rgb]{0.56,0.35,0.01}{\textit{#1}}}
\newcommand{\RegionMarkerTok}[1]{#1}
\newcommand{\SpecialCharTok}[1]{\textcolor[rgb]{0.00,0.00,0.00}{#1}}
\newcommand{\SpecialStringTok}[1]{\textcolor[rgb]{0.31,0.60,0.02}{#1}}
\newcommand{\StringTok}[1]{\textcolor[rgb]{0.31,0.60,0.02}{#1}}
\newcommand{\VariableTok}[1]{\textcolor[rgb]{0.00,0.00,0.00}{#1}}
\newcommand{\VerbatimStringTok}[1]{\textcolor[rgb]{0.31,0.60,0.02}{#1}}
\newcommand{\WarningTok}[1]{\textcolor[rgb]{0.56,0.35,0.01}{\textbf{\textit{#1}}}}
\usepackage{graphicx,grffile}
\makeatletter
\def\maxwidth{\ifdim\Gin@nat@width>\linewidth\linewidth\else\Gin@nat@width\fi}
\def\maxheight{\ifdim\Gin@nat@height>\textheight\textheight\else\Gin@nat@height\fi}
\makeatother
% Scale images if necessary, so that they will not overflow the page
% margins by default, and it is still possible to overwrite the defaults
% using explicit options in \includegraphics[width, height, ...]{}
\setkeys{Gin}{width=\maxwidth,height=\maxheight,keepaspectratio}
% Set default figure placement to htbp
\makeatletter
\def\fps@figure{htbp}
\makeatother
\setlength{\emergencystretch}{3em} % prevent overfull lines
\providecommand{\tightlist}{%
  \setlength{\itemsep}{0pt}\setlength{\parskip}{0pt}}
\setcounter{secnumdepth}{5}
\usepackage{booktabs}
\usepackage{setspace}
\doublespacing
\usepackage[labelfont=bf]{caption}
\usepackage{booktabs}
\usepackage{longtable}
\usepackage{array}
\usepackage{multirow}
\usepackage{wrapfig}
\usepackage{float}
\usepackage{colortbl}
\usepackage{pdflscape}
\usepackage{tabu}
\usepackage{threeparttable}
\usepackage{threeparttablex}
\usepackage[normalem]{ulem}
\usepackage{makecell}
\usepackage{xcolor}

\title{The Influence of Individual Characterisitcs on Public Transportation
Planning\thanks{xx}}
\author{Iris Zhong}
\date{}

\begin{document}
\maketitle
\begin{abstract}
xx
\end{abstract}

\begin{verbatim}
## Warning: package 'tidyverse' was built under R version 3.5.3
\end{verbatim}

\begin{verbatim}
## Warning: package 'ggplot2' was built under R version 3.5.3
\end{verbatim}

\begin{verbatim}
## Warning: package 'tibble' was built under R version 3.5.3
\end{verbatim}

\begin{verbatim}
## Warning: package 'tidyr' was built under R version 3.5.3
\end{verbatim}

\begin{verbatim}
## Warning: package 'readr' was built under R version 3.5.3
\end{verbatim}

\begin{verbatim}
## Warning: package 'purrr' was built under R version 3.5.3
\end{verbatim}

\begin{verbatim}
## Warning: package 'dplyr' was built under R version 3.5.3
\end{verbatim}

\begin{verbatim}
## Warning: package 'stringr' was built under R version 3.5.3
\end{verbatim}

\begin{verbatim}
## Warning: package 'forcats' was built under R version 3.5.3
\end{verbatim}

\begin{verbatim}
## Warning: package 'lubridate' was built under R version 3.5.3
\end{verbatim}

\begin{verbatim}
## Warning: package 'stargazer' was built under R version 3.5.2
\end{verbatim}

\begin{verbatim}
## Warning: package 'corrplot' was built under R version 3.5.3
\end{verbatim}

\begin{verbatim}
## Warning: package 'Hmisc' was built under R version 3.5.3
\end{verbatim}

\begin{verbatim}
## Warning: package 'survival' was built under R version 3.5.3
\end{verbatim}

\begin{verbatim}
## Warning: package 'Formula' was built under R version 3.5.2
\end{verbatim}

\begin{verbatim}
## Warning: package 'dlookr' was built under R version 3.5.3
\end{verbatim}

\begin{verbatim}
## Warning: package 'mice' was built under R version 3.5.3
\end{verbatim}

\hypertarget{literature-review}{%
\section{Literature Review}\label{literature-review}}

Allen et al.~(2016) study the reasoning of the failure of a referendum
on a congestion charging scheme in Edinburgh. Instead of using direct
voting data, they conduct a survey after the referendum, which allows
them to ask more specific questions. Researchers can gain detailed data
by surveying, because the unit of measurement is each individual;
however, a possible disadvantage of surveying is that respondents who
turn in the questionnaire tend to have stronger attitudes towards the
proposal, generating sampling bias. They conclude that people who use
cars as the primary transportation mean, demonstrate a misconception of
the pricing plan, or question the effectiveness of the scheme at
reducing congestion are more likely to oppose it. Their findings can
give insights to the similar failure in the Gwinnett referendum. Voters
against the proposal could be those who rarely use public transportation
and those who are not convinced by the effectiveness of expanding public
transit in alleviating the traffic.

Another crucial factor is the accessibility of the proposed transit
system. Kinsey et al.~(2010) examine the relationship between the
distance to the scheduled railway station and voter turnout by studying
the Seattle monorail referendum. They introduce the concept of diffused
and concentrated benefit/cost. People who live far from the monorail
enjoy the diffused benefit of less traffic congestion, and bear the
diffused cost of increased tax. People living close to the rail
experience the same diffused benefit and cost, but they also gain the
concentrated benefit of easily accessing the public good. Finally, those
who live very close to the railway have the same benefits and costs, but
they also face the concentrated cost such as inconvenience during
construction. Since ``people are more strongly motivated to avoid losses
than to approach gains,'' they expect a higher turnout rate in farther
places with votes for ``no,'' which is verified from their analyses.
Besides distance, they also find out precincts with a higher percentage
of people of lower socioeconomic status or young people have a lower
turnout rate. Interestingly, there is a significant interaction between
partisanship and distance, which would be also tested in my study. In
essence, the effect of distance on turnout is weakened by partisanship,
and vanishes beyond a threshold of distance. Even though my dependent
variable is voters' responses rather than turnout, it can be inferred
from Kinsey et al.'s findings that people farther away from the transit
system would vote against the referendum more. However, the relationship
might be non-linear and requires some form of transformation. Regarding
the methods, they utilize the spatial lag model to correct for
autocorrelation, which is proper to use in my project as well since both
studies use precinct-level data.

\hypertarget{background}{%
\section{Background}\label{background}}

current transportation future plan referendum

\hypertarget{data-methods}{%
\section{Data \& Methods}\label{data-methods}}

\hypertarget{conceptual-model}{%
\subsection{Conceptual model}\label{conceptual-model}}

According to previous research, sociodemographic elements can influence
people's voting decisions in the referendum. For example, the effect of
income is mixed: on the one hand, people with higher income will pay a
smaller portion of their earnings for the implementation of the plan; on
the other hand, they will pay a larger amount of tax. Bollino (2008)
finds a positive correlation between income and people's willingness to
pay for renewable resources. Burkhardt and Chan (2017) separate the
influence of income from tax, and discover their opposite effects on
voting. Therefore, it is worth considering the relationship between
income and percentage of supporters in this referendum. Voters'
partisanship attachment is found to be a significant factor as well in
Burkhardt and Chan's (2017) paper. Areas with higher proportions of
Republicans are less supportive of fiscally costly propositions. In my
project, it can be hypothesized that tracts that have a higher
proportion of Trump supporters tend to have a lower percentage of
agreement to the proposal.

In addition, some factors related to transportation can intuitively
shape people's attitudes towards public transit. For example, the areas
in which people do not use public transit at all might have a higher
percentage of refusal of the proposal. People who have to travel a long
time to work are more likely to support the extension plan if it helps
save time. These two factors serve as controls in the models.

Finally, people favor the proposition if it benefits them. Specifically,
tracts that are not covered by public transport at present but will be
covered in the expansion plan are predicted to support the proposal
more.

\hypertarget{data}{%
\subsection{Data}\label{data}}

The ballot results of the Gwinnett County referendum on Mar 19th, 2019
is obtained from a website powered by Scytl, a trusted source of
election outcomes. The cross-sectional dataset contains the voting
information of all 157 precincts in the county. The dependent variable
-- the proportion of supporters of the referendum, and the voter turnout
rate are calculated from this data source.

The result of the 2016 Presidential Election is chosen to reflect
partisanship. A cross-sectional precinct-level election data is obtained
from the MIT Election Data and Science Lab website. The number of votes
for Trump at each precinct in Gwinnett county comes from this dataset.

Next, cross-sectional sociodemographic characteristics at the census
tract level are found in Census Bureau via an R package called
\texttt{tidycensus}. The data comes from the American Community Survey
5-year estimate published in 2018. The median age and median income are
collected here. In addition, the proportion of white, the percentage of
people who go to work by public transportation, and the percentage of
people who travel more than an hour to work are calculated by dividing
the relevant variables by the total population or the survey sample
size.

The information on whether the tract enjoys the proximity of public
transportation now and future can be acquired from spatial maps and
analyses. First, a precinct-level shapefile of Gwinnett County made in
2018 is obtained from the Georgia General Assembly. Gwinnett County maps
with current and proposed future public transit systems are available on
the Gwinnett County government website. I select the short-range
expansion plan (Y2020 -- 2025) because the cost and benefit of the
expansion in the far future are discounted more. After modifications in
QGIS software, the maps are then transformed into spatial data readable
in R. Five-hundred-meter buffer zones are created around bus and railway
routes. The categorical variable \emph{current\_plan} has four levels: 1
represents no transportation near this tract at present and in the
future; 2 represents the tracts that are accessible to public transit
currently but not in the future; 3 stands for the tracts that do not
have transit at present but will do in the expansion plan; 4 represents
the tracts that have and will have public transit for now and for the
future.

As noted above, both referendum and 2016 election data are collected at
the precinct level. However, the other datasets are performed at the
census tract level. Therefore, referendum and election data are
redistributed by the areas shared by the precinct and the tract. See the
data appendix for detailed steps of transformation.

The final dataset joins the datasets above by GEOID. It is
cross-sectional, measured with the unit of the census tract. A
description of the variables can be found in \emph{Table 1}.

\begin{table}
\centering
\caption{Variable definitions}
\label{variableDefinitions}
\begin{tabular}{ll}
\hline
\hline
Variable name      & Description                                   \\
\hline
GEOID          & The geographic identifier of the census tract                 \\
medage        & The median age of the population in the tract               \\
medincome        & The median income of the population in the tract               \\
white\_pct           & The percentage of white population in the tract      \\
public\_pct          & The percentage of people who go to work by public transportation (excluding taxi or cab)      \\
time\_pct           & The percentage of people who travel more than an hour to work      \\
trump\_pct          & The estimated percentage of votes for Donald Trump in that tract      \\
voter\_turnout  & The estimated percentage of voters who voted in this referendum in the tract \\
yes\_pct           & The estimated percentage of voters who voted yes in this referendum in the tract      \\
plan\_yes          & Whether the tract is covered by the public transportation now and in the short-range, \\ & defined by whether any transportation is available within 500 meters. \\ & 1 stands for the tract doesn't have transit both now and in the short-range plan. \\ & 2 stands for the tract has transit now but not in the short-range plan. \\ & 3 stands for the tract that doesn't have transit now and will have in the future.\\ & 4 stands for the tract that has public transit both now and in the future      \\

\hline
\end{tabular}
\end{table}

\begin{table}[!htbp] \centering 
  \caption{Summary statistics} 
  \label{summaryStats} 
\begin{tabular}{@{\extracolsep{5pt}}lccccccc} 
\\[-1.8ex]\hline 
\hline \\[-1.8ex] 
Statistic & \multicolumn{1}{c}{N} & \multicolumn{1}{c}{Mean} & \multicolumn{1}{c}{St. Dev.} & \multicolumn{1}{c}{Min} & \multicolumn{1}{c}{Pctl(25)} & \multicolumn{1}{c}{Pctl(75)} & \multicolumn{1}{c}{Max} \\ 
\hline \\[-1.8ex] 
medage & 113 & 35.56 & 4.58 & 26 & 32.8 & 38.8 & 52 \\ 
medincome & 113 & 69,439.24 & 24,358.44 & 33,020 & 51,429 & 82,845 & 156,136 \\ 
white\_pct & 113 & 0.48 & 0.15 & 0.17 & 0.38 & 0.61 & 0.89 \\ 
public\_pct & 113 & 0.01 & 0.01 & 0 & 0.002 & 0.02 & 0 \\ 
time\_pct & 113 & 0.16 & 0.05 & 0.04 & 0.12 & 0.20 & 0.31 \\ 
trump\_pct & 113 & 0.40 & 0.15 & 0.11 & 0.27 & 0.52 & 0.69 \\ 
voter\_turnout & 113 & 0.16 & 0.06 & 0.05 & 0.13 & 0.18 & 0.37 \\ 
yes\_pct & 113 & 0.53 & 0.14 & 0.27 & 0.42 & 0.61 & 0.84 \\ 
\hline \\[-1.8ex] 
\end{tabular} 
\end{table}

\begin{Shaded}
\begin{Highlighting}[]
\NormalTok{data_numeric <-}\StringTok{ }\NormalTok{final_data }\OperatorTok
\StringTok{   }\KeywordTok{select}\NormalTok{(}\OperatorTok{-}\KeywordTok{c}\NormalTok{(current_plan,GEOID))}

\NormalTok{data_cor =}\StringTok{ }\KeywordTok{cor}\NormalTok{(data_numeric)}

\NormalTok{data_cor_}\DecValTok{1}\NormalTok{ <-}\StringTok{ }\KeywordTok{rcorr}\NormalTok{(}\KeywordTok{as.matrix}\NormalTok{(data_numeric))}
\NormalTok{M <-}\StringTok{ }\NormalTok{data_cor_}\DecValTok{1}\OperatorTok{$}\NormalTok{r}
\NormalTok{p_mat <-}\StringTok{ }\NormalTok{data_cor_}\DecValTok{1}\OperatorTok{$}\NormalTok{P}
\KeywordTok{corrplot}\NormalTok{(M, }\DataTypeTok{type =} \StringTok{"upper"}\NormalTok{, }\DataTypeTok{order =} \StringTok{"hclust"}\NormalTok{, }
         \DataTypeTok{p.mat =}\NormalTok{ p_mat, }\DataTypeTok{sig.level =} \FloatTok{0.05}\NormalTok{)}
\end{Highlighting}
\end{Shaded}

\includegraphics{Zhong_paper_files/figure-latex/create correlation matrix-1.pdf}

\emph{Table 2} lists the summary statistics. Median income is skewed to
the right, with a few tracts demonstrating high income levels. Such a
pattern of inequality is universally observed.

The usage of public transit is low. On average, only one percent of the
population relies on public transportation to go to work. As proved by
later analyses, the distribution is highly skewed, and will be
transformed in some models. \textbf{public\_pct max needs to be fixed}

The variable \emph{current\_plan} is not in \emph{Table 2} because it is
a categorical variable. In sum, 31 tracts do not have access to public
transport within 500 meters, both now and in the short-range. There is
one tract that is categorized as 2, indicating that it has public
transit at present, but not in the short-range proposition. It could be
due to the plan of reducing circuitous routing. Twenty-two tracts do not
enjoy the proximity of public transport now, but will in the
short-range. Lastly, 59 tracts have public transit available both at
present and in the short-range plan. Since a level of this variable
contains only one value (``2''), problems with degrees of freedom will
potentially arise.

A correlation matrix is created from \texttt{corrplot} package to
investigate the correlation between each pair of factors (see
\emph{Figure ???}). Positive correlations are in blue colors, and
negative correlations are in red colors. The magnitude is reflected by
the color intensity and the size of the dot. Non-significant (p
\textgreater{} 0.05) correlations are marked with a cross.
\emph{Current\_plan} is omitted in the matrix because it is categorical.
The dependent variable \emph{yes\_pct} is significantly correlated with
every independent variable. \emph{Medage}, \emph{voter\_turnout},
\emph{medincome}, \emph{white\_pct}, and \emph{trump\_pct} are all
strongly positively correlated with each other, and all of them are
negatively associated with \emph{yes\_pct}. Given the number of strong,
significant correlations among the factors, it is essential to check
collinearity in the regression model.

The major limitation of the data is the unit conversion from precinct to
census tract. Such a method assumes that residents in one precinct have
the same characteristics, and population density is identical. Clearly,
the assumptions cannot be satisfied in a real dataset.

\hypertarget{model-specification}{%
\subsection{Model specification}\label{model-specification}}

Four models are tested in the analysis. The dependent variable is
\emph{yes\_pct}, the proportion of supporters of the referendum in a
tract. Independent variables include \emph{medincome},
\emph{white\_pct}, \emph{trump\_pct}, and \emph{current\_plan}. For more
information about these variables, refer to \emph{Table ???}.

Model 1 adopts linear regression model with all the original variables:

\(yes\_pct_t = \beta_0+\beta_1medincome_t+\beta_2white\_pct_t+\beta_3trump\_pct_t+\beta_4current\_plan_t+\beta_5medage_t+\beta_6public\_pct_t+\beta_7time\_pct_t+\beta_8voter\_turnout_t+\epsilon_t\)

xxxxx

Model 2 is identical to Model 1, except that it uses logistic
regression. Even though the dependent variable is not binary, previous
research has used logistic regression in a similar scenario (???).

\(logit(yes\_pct_t) = \beta_0+\beta_1medincome_t+\beta_2white\_pct_t+\beta_3trump\_pct_t+\beta_4current\_plan_t+\beta_5medage_t+\beta_6public\_pct_t+\beta_7time\_pct_t+\beta_8voter\_turnout_t+\epsilon_4\)

\emph{Medincome}, \emph{public\_pct} and \emph{voter\_turnout} are found
to be skewed. Thus, Model 3 uses multiple linear regression after log or
square root transformation on these three variables.

\(yes\_pct_t = \beta_0+\beta_1log(medincome_t)+\beta_2white\_pct_t+\beta_3trump\_pct_t+\beta_4current\_plan_t+\beta_5medage_t+\beta_6log(public\_pct_t + 0,01)+\beta_7time\_pct_t+\beta_8\sqrt{voter\_turnout_t}+\epsilon_t\)

Finally, Model 4 adopts the same transformation as Model 3, with
logistic regression.

\(logit(yes\_pct_t) = \beta_0+\beta_1log(medincome_t)+\beta_2white\_pct_t+\beta_3trump\_pct_t+\beta_4current\_plan_t+\beta_5medage_t+\beta_6log(public\_pct_t + 0.01)+\beta_7time\_pct_t+\beta_8\sqrt{voter\_turnout_t}+\epsilon_t\)

\hypertarget{results-discussion}{%
\section{Results \& Discussion}\label{results-discussion}}

model 1: linear

\begin{Shaded}
\begin{Highlighting}[]
\NormalTok{mod1 <-}\StringTok{ }\KeywordTok{lm}\NormalTok{(}\DataTypeTok{data =}\NormalTok{ final_data, yes_pct }\OperatorTok{~}\StringTok{ }\NormalTok{medage }\OperatorTok{+}\StringTok{ }\NormalTok{medincome }\OperatorTok{+}\StringTok{ }\NormalTok{white_pct }
           \OperatorTok{+}\StringTok{ }\NormalTok{public_pct }\OperatorTok{+}\StringTok{ }\NormalTok{time_pct }\OperatorTok{+}\StringTok{ }\NormalTok{trump_pct }\OperatorTok{+}\StringTok{ }\NormalTok{voter_turnout }\OperatorTok{+}
\StringTok{              }\NormalTok{current_plan)}
\KeywordTok{summary}\NormalTok{(mod1)}
\end{Highlighting}
\end{Shaded}

Call: lm(formula = yes\_pct \textasciitilde{} medage + medincome +
white\_pct + public\_pct + time\_pct + trump\_pct + voter\_turnout +
current\_plan, data = final\_data)

Residuals: Min 1Q Median 3Q Max -0.103714 -0.032077 -0.002399 0.030226
0.101740

Coefficients: Estimate Std. Error t value
Pr(\textgreater\textbar t\textbar)\\
(Intercept) 8.041e-01 4.707e-02 17.085 \textless{} 2e-16 \textbf{\emph{
medage 2.738e-03 1.481e-03 1.849 0.06739 .\\
medincome 1.326e-07 2.930e-07 0.453 0.65183\\
white\_pct 1.143e-01 5.326e-02 2.147 0.03420 }\\
public\_pct 8.173e-01 3.382e-01 2.417 0.01743 *\\
time\_pct -3.057e-01 9.048e-02 -3.378 0.00103 } trump\_pct -8.859e-01
5.709e-02 -15.517 \textless{} 2e-16 \textbf{\emph{ voter\_turnout
-3.257e-01 1.262e-01 -2.582 0.01125 }\\
current\_plan2 1.888e-02 4.399e-02 0.429 0.66866\\
current\_plan3 -3.176e-02 1.201e-02 -2.645 0.00945 } current\_plan4
1.612e-02 1.155e-02 1.395 0.16606\\
--- Signif. codes: 0 `\emph{\textbf{' 0.001 '}' 0.01 '}' 0.05 `.' 0.1 '
' 1

Residual standard error: 0.04197 on 102 degrees of freedom Multiple
R-squared: 0.9174, Adjusted R-squared: 0.9093 F-statistic: 113.2 on 10
and 102 DF, p-value: \textless{} 2.2e-16

model 1 assumption checking

\begin{Shaded}
\begin{Highlighting}[]
\KeywordTok{plot}\NormalTok{(}\KeywordTok{predict}\NormalTok{(mod1),}\KeywordTok{resid}\NormalTok{(mod1),}\DataTypeTok{col=}\StringTok{"midnightblue"}\NormalTok{,}\DataTypeTok{pch=}\DecValTok{18}\NormalTok{,}\DataTypeTok{main=}\StringTok{"Residual plot - Model 1"}\NormalTok{)}
\KeywordTok{abline}\NormalTok{(}\DecValTok{0}\NormalTok{,}\DecValTok{0}\NormalTok{,}\DataTypeTok{col=}\StringTok{"red"}\NormalTok{)}
\end{Highlighting}
\end{Shaded}

\includegraphics{Zhong_paper_files/figure-latex/unnamed-chunk-3-1.pdf}

collinearity:

\begin{Shaded}
\begin{Highlighting}[]
\KeywordTok{library}\NormalTok{(car)}
\end{Highlighting}
\end{Shaded}

\begin{verbatim}
## Warning: package 'car' was built under R version 3.5.3
\end{verbatim}

\begin{verbatim}
## Loading required package: carData
\end{verbatim}

\begin{verbatim}
## Warning: package 'carData' was built under R version 3.5.3
\end{verbatim}

\begin{verbatim}
## 
## Attaching package: 'car'
\end{verbatim}

\begin{verbatim}
## The following object is masked from 'package:dplyr':
## 
##     recode
\end{verbatim}

\begin{verbatim}
## The following object is masked from 'package:purrr':
## 
##     some
\end{verbatim}

\begin{Shaded}
\begin{Highlighting}[]
\KeywordTok{vif}\NormalTok{(mod1)}
\end{Highlighting}
\end{Shaded}

\begin{verbatim}
              GVIF Df GVIF^(1/(2*Df))
\end{verbatim}

medage 2.921005 1 1.709095 medincome 3.237296 1 1.799249 white\_pct
3.900578 1 1.974988 public\_pct 1.163004 1 1.078427 time\_pct 1.563279 1
1.250312 trump\_pct 4.533128 1 2.129114 voter\_turnout 3.372750 1
1.836505 current\_plan 1.853504 3 1.108321

all below 5: good, no collinearity problem

model 2: no interaction, logistic

\begin{Shaded}
\begin{Highlighting}[]
\NormalTok{mod2 <-}\StringTok{ }\KeywordTok{glm}\NormalTok{(}\DataTypeTok{data =}\NormalTok{ final_data, yes_pct }\OperatorTok{~}\StringTok{ }\NormalTok{medage }\OperatorTok{+}\StringTok{ }\NormalTok{medincome }\OperatorTok{+}\StringTok{ }\NormalTok{white_pct }\OperatorTok{+}\StringTok{ }\NormalTok{public_pct }\OperatorTok{+}\StringTok{ }\NormalTok{time_pct }\OperatorTok{+}\StringTok{ }\NormalTok{trump_pct }\OperatorTok{+}\StringTok{ }\NormalTok{voter_turnout }\OperatorTok{+}\StringTok{ }\NormalTok{current_plan, }\DataTypeTok{family =} \StringTok{"binomial"}\NormalTok{)}
\end{Highlighting}
\end{Shaded}

\begin{verbatim}
## Warning in eval(family$initialize): non-integer #successes in a binomial glm!
\end{verbatim}

\begin{Shaded}
\begin{Highlighting}[]
\KeywordTok{summary}\NormalTok{(mod2)}
\end{Highlighting}
\end{Shaded}

Call: glm(formula = yes\_pct \textasciitilde{} medage + medincome +
white\_pct + public\_pct + time\_pct + trump\_pct + voter\_turnout +
current\_plan, family = ``binomial'', data = final\_data)

Deviance Residuals: Min 1Q Median 3Q Max\\
-0.224855 -0.066351 -0.001643 0.067049 0.208799

Coefficients: Estimate Std. Error z value
Pr(\textgreater\textbar z\textbar) (Intercept) 1.328e+00 2.320e+00 0.572
0.567 medage 1.070e-02 7.223e-02 0.148 0.882 medincome 6.272e-07
1.435e-05 0.044 0.965 white\_pct 5.506e-01 2.639e+00 0.209 0.835
public\_pct 3.528e+00 1.715e+01 0.206 0.837 time\_pct -1.390e+00
4.468e+00 -0.311 0.756 trump\_pct -3.779e+00 2.844e+00 -1.329 0.184
voter\_turnout -1.441e+00 6.219e+00 -0.232 0.817 current\_plan2
5.722e-02 2.128e+00 0.027 0.979 current\_plan3 -1.378e-01 5.889e-01
-0.234 0.815 current\_plan4 5.908e-02 5.632e-01 0.105 0.916

(Dispersion parameter for binomial family taken to be 1)

\begin{verbatim}
Null deviance: 9.05063  on 112  degrees of freedom
\end{verbatim}

Residual deviance: 0.82869 on 102 degrees of freedom AIC: 135.96

Number of Fisher Scoring iterations: 4

model 1 \& 2 table

\begin{table}[!htbp] \centering 
  \caption{Initial regression results} 
  \label{initialResults} 
\begin{tabular}{@{\extracolsep{5pt}}lcc} 
\\[-1.8ex]\hline 
\hline \\[-1.8ex] 
 & \multicolumn{2}{c}{\textit{Dependent variable:}} \\ 
\cline{2-3} 
\\[-1.8ex] & \multicolumn{2}{c}{yes\_pct} \\ 
\\[-1.8ex] & \textit{OLS} & \textit{logistic} \\ 
\\[-1.8ex] & (1) & (2)\\ 
\hline \\[-1.8ex] 
 medage & 0.003$^{*}$ & 0.011 \\ 
  & (0.001) & (0.072) \\ 
  & & \\ 
 medincome & 0.00000 & 0.00000 \\ 
  & (0.00000) & (0.00001) \\ 
  & & \\ 
 white\_pct & 0.114$^{**}$ & 0.551 \\ 
  & (0.053) & (2.639) \\ 
  & & \\ 
 public\_pct & 0.817$^{**}$ & 3.528 \\ 
  & (0.338) & (17.153) \\ 
  & & \\ 
 time\_pct & $-$0.306$^{***}$ & $-$1.390 \\ 
  & (0.090) & (4.468) \\ 
  & & \\ 
 trump\_pct & $-$0.886$^{***}$ & $-$3.779 \\ 
  & (0.057) & (2.844) \\ 
  & & \\ 
 voter\_turnout & $-$0.326$^{**}$ & $-$1.441 \\ 
  & (0.126) & (6.219) \\ 
  & & \\ 
 current\_plan2 & 0.019 & 0.057 \\ 
  & (0.044) & (2.128) \\ 
  & & \\ 
 current\_plan3 & $-$0.032$^{***}$ & $-$0.138 \\ 
  & (0.012) & (0.589) \\ 
  & & \\ 
 current\_plan4 & 0.016 & 0.059 \\ 
  & (0.012) & (0.563) \\ 
  & & \\ 
 Constant & 0.804$^{***}$ & 1.328 \\ 
  & (0.047) & (2.320) \\ 
  & & \\ 
\hline \\[-1.8ex] 
Observations & 113 & 113 \\ 
R$^{2}$ & 0.917 &  \\ 
Adjusted R$^{2}$ & 0.909 &  \\ 
Log Likelihood &  & $-$56.978 \\ 
Akaike Inf. Crit. &  & 135.955 \\ 
Residual Std. Error & 0.042 (df = 102) &  \\ 
F Statistic & 113.224$^{***}$ (df = 10; 102) &  \\ 
\hline 
\hline \\[-1.8ex] 
\textit{Note:}  & \multicolumn{2}{l}{$^{*}$p$<$0.1; $^{**}$p$<$0.05; $^{***}$p$<$0.01} \\ 
 & \multicolumn{2}{l}{Initial linear and logistic regression results} \\ 
\end{tabular} 
\end{table}

model 3: no interaction, some transformations, linear

step 1: find the skewed variables

medincome, public\_pct, voter\_turnout

step 2: transform them

\begin{Shaded}
\begin{Highlighting}[]
\NormalTok{data_tf <-}\StringTok{ }\NormalTok{final_data }\OperatorTok
\StringTok{   }\KeywordTok{mutate}\NormalTok{(}\DataTypeTok{log_medincome =} \KeywordTok{log}\NormalTok{(medincome),}
          \DataTypeTok{log_public_pct =} \KeywordTok{log}\NormalTok{(public_pct }\OperatorTok{+}\StringTok{ }\FloatTok{0.01}\NormalTok{),}
          \DataTypeTok{sqrt_voter_turnout =}\NormalTok{ (voter_turnout)}\OperatorTok{^}\FloatTok{0.5}\NormalTok{) }\OperatorTok
\StringTok{   }\KeywordTok{select}\NormalTok{(}\OperatorTok{-}\KeywordTok{c}\NormalTok{(medincome, public_pct, voter_turnout))}
\KeywordTok{find_skewness}\NormalTok{(data_tf)}
\end{Highlighting}
\end{Shaded}

integer(0)

step 3: model them

\begin{Shaded}
\begin{Highlighting}[]
\NormalTok{mod3 <-}\StringTok{ }\KeywordTok{lm}\NormalTok{(}\DataTypeTok{data =}\NormalTok{ data_tf, yes_pct }\OperatorTok{~}\StringTok{ }\NormalTok{medage }\OperatorTok{+}\StringTok{ }\NormalTok{log_medincome }\OperatorTok{+}\StringTok{ }\NormalTok{white_pct }\OperatorTok{+}\StringTok{ }\NormalTok{log_public_pct }\OperatorTok{+}\StringTok{ }\NormalTok{time_pct }\OperatorTok{+}\StringTok{ }\NormalTok{trump_pct }\OperatorTok{+}\StringTok{ }\NormalTok{sqrt_voter_turnout }\OperatorTok{+}\StringTok{ }\NormalTok{current_plan)}
\KeywordTok{summary}\NormalTok{(mod3)}
\end{Highlighting}
\end{Shaded}

Call: lm(formula = yes\_pct \textasciitilde{} medage + log\_medincome +
white\_pct + log\_public\_pct + time\_pct + trump\_pct +
sqrt\_voter\_turnout + current\_plan, data = data\_tf)

Residuals: Min 1Q Median 3Q Max -0.099920 -0.031591 -0.003479 0.027983
0.098568

Coefficients: Estimate Std. Error t value
Pr(\textgreater\textbar t\textbar)\\
(Intercept) 1.031840 0.218778 4.716 7.64e-06 \textbf{\emph{ medage
0.003603 0.001511 2.384 0.01899 }\\
log\_medincome -0.008331 0.022302 -0.374 0.70952\\
white\_pct 0.116703 0.050746 2.300 0.02350 *\\
log\_public\_pct 0.022304 0.008263 2.699 0.00814 } time\_pct -0.281374
0.089698 -3.137 0.00223 ** trump\_pct -0.856963 0.058590 -14.626
\textless{} 2e-16 *\textbf{ sqrt\_voter\_turnout -0.319107 0.103460
-3.084 0.00263 } current\_plan2 0.021184 0.043430 0.488 0.62677\\
current\_plan3 -0.033367 0.011777 -2.833 0.00555 ** current\_plan4
0.013583 0.011459 1.185 0.23864\\
--- Signif. codes: 0 `\emph{\textbf{' 0.001 '}' 0.01 '}' 0.05 `.' 0.1 '
' 1

Residual standard error: 0.04125 on 102 degrees of freedom Multiple
R-squared: 0.9202, Adjusted R-squared: 0.9123 F-statistic: 117.6 on 10
and 102 DF, p-value: \textless{} 2.2e-16

model 3 assumption checking

\begin{Shaded}
\begin{Highlighting}[]
\KeywordTok{plot}\NormalTok{(}\KeywordTok{predict}\NormalTok{(mod3),}\KeywordTok{resid}\NormalTok{(mod3),}\DataTypeTok{col=}\StringTok{"midnightblue"}\NormalTok{,}\DataTypeTok{pch=}\DecValTok{18}\NormalTok{,}\DataTypeTok{main=}\StringTok{"Residual plot - Model 3"}\NormalTok{)}
\KeywordTok{abline}\NormalTok{(}\DecValTok{0}\NormalTok{,}\DecValTok{0}\NormalTok{,}\DataTypeTok{col=}\StringTok{"red"}\NormalTok{)}
\end{Highlighting}
\end{Shaded}

\includegraphics{Zhong_paper_files/figure-latex/unnamed-chunk-8-1.pdf}

collinearity:

\begin{Shaded}
\begin{Highlighting}[]
\KeywordTok{vif}\NormalTok{(mod3)}
\end{Highlighting}
\end{Shaded}

\begin{verbatim}
                   GVIF Df GVIF^(1/(2*Df))
\end{verbatim}

medage 3.149713 1 1.774743 log\_medincome 3.738783 1 1.933593 white\_pct
3.666112 1 1.914709 log\_public\_pct 1.149252 1 1.072032 time\_pct
1.590210 1 1.261035 trump\_pct 4.942057 1 2.223074 sqrt\_voter\_turnout
3.628296 1 1.904809 current\_plan 1.883541 3 1.111295

all below 5: no collinearity

model 4: transformation, logistic

\begin{Shaded}
\begin{Highlighting}[]
\NormalTok{mod4 <-}\StringTok{ }\KeywordTok{glm}\NormalTok{(}\DataTypeTok{data =}\NormalTok{ data_tf, yes_pct }\OperatorTok{~}\StringTok{ }\NormalTok{medage }\OperatorTok{+}\StringTok{ }\NormalTok{log_medincome }\OperatorTok{+}\StringTok{ }\NormalTok{white_pct }
           \OperatorTok{+}\StringTok{ }\NormalTok{log_public_pct }\OperatorTok{+}\StringTok{ }\NormalTok{time_pct }\OperatorTok{+}\StringTok{ }\NormalTok{trump_pct }\OperatorTok{+}\StringTok{ }\NormalTok{sqrt_voter_turnout }\OperatorTok{+}
\StringTok{              }\NormalTok{current_plan, }\DataTypeTok{family =} \StringTok{"binomial"}\NormalTok{)}
\end{Highlighting}
\end{Shaded}

\begin{verbatim}
## Warning in eval(family$initialize): non-integer #successes in a binomial glm!
\end{verbatim}

\begin{Shaded}
\begin{Highlighting}[]
\KeywordTok{summary}\NormalTok{(mod4)}
\end{Highlighting}
\end{Shaded}

Call: glm(formula = yes\_pct \textasciitilde{} medage + log\_medincome +
white\_pct + log\_public\_pct + time\_pct + trump\_pct +
sqrt\_voter\_turnout + current\_plan, family = ``binomial'', data =
data\_tf)

Deviance Residuals: Min 1Q Median 3Q Max\\
-0.21662 -0.06950 0.00000 0.06082 0.20908

Coefficients: Estimate Std. Error z value
Pr(\textgreater\textbar z\textbar) (Intercept) 2.30500 10.89739 0.212
0.832 medage 0.01489 0.07514 0.198 0.843 log\_medincome -0.03485 1.11172
-0.031 0.975 white\_pct 0.56326 2.55987 0.220 0.826 log\_public\_pct
0.09511 0.41908 0.227 0.820 time\_pct -1.28743 4.50172 -0.286 0.775
trump\_pct -3.64687 2.95956 -1.232 0.218 sqrt\_voter\_turnout -1.45781
5.21572 -0.280 0.780 current\_plan2 0.06583 2.13728 0.031 0.975
current\_plan3 -0.14496 0.58781 -0.247 0.805 current\_plan4 0.04787
0.56805 0.084 0.933

(Dispersion parameter for binomial family taken to be 1)

\begin{verbatim}
Null deviance: 9.0506  on 112  degrees of freedom
\end{verbatim}

Residual deviance: 0.7965 on 102 degrees of freedom AIC: 136.15

Number of Fisher Scoring iterations: 4

model 3 \& 4 table

\begin{table}[!htbp] \centering 
  \caption{Data-transformed regression results} 
  \label{transformedResults} 
\begin{tabular}{@{\extracolsep{5pt}}lcc} 
\\[-1.8ex]\hline 
\hline \\[-1.8ex] 
 & \multicolumn{2}{c}{\textit{Dependent variable:}} \\ 
\cline{2-3} 
\\[-1.8ex] & \multicolumn{2}{c}{yes\_pct} \\ 
\\[-1.8ex] & \textit{OLS} & \textit{logistic} \\ 
\\[-1.8ex] & (1) & (2)\\ 
\hline \\[-1.8ex] 
 medage & 0.004$^{**}$ & 0.015 \\ 
  & (0.002) & (0.075) \\ 
  & & \\ 
 log\_medincome & $-$0.008 & $-$0.035 \\ 
  & (0.022) & (1.112) \\ 
  & & \\ 
 white\_pct & 0.117$^{**}$ & 0.563 \\ 
  & (0.051) & (2.560) \\ 
  & & \\ 
 log\_public\_pct & 0.022$^{***}$ & 0.095 \\ 
  & (0.008) & (0.419) \\ 
  & & \\ 
 time\_pct & $-$0.281$^{***}$ & $-$1.287 \\ 
  & (0.090) & (4.502) \\ 
  & & \\ 
 trump\_pct & $-$0.857$^{***}$ & $-$3.647 \\ 
  & (0.059) & (2.960) \\ 
  & & \\ 
 sqrt\_voter\_turnout & $-$0.319$^{***}$ & $-$1.458 \\ 
  & (0.103) & (5.216) \\ 
  & & \\ 
 current\_plan2 & 0.021 & 0.066 \\ 
  & (0.043) & (2.137) \\ 
  & & \\ 
 current\_plan3 & $-$0.033$^{***}$ & $-$0.145 \\ 
  & (0.012) & (0.588) \\ 
  & & \\ 
 current\_plan4 & 0.014 & 0.048 \\ 
  & (0.011) & (0.568) \\ 
  & & \\ 
 Constant & 1.032$^{***}$ & 2.305 \\ 
  & (0.219) & (10.897) \\ 
  & & \\ 
\hline \\[-1.8ex] 
Observations & 113 & 113 \\ 
R$^{2}$ & 0.920 &  \\ 
Adjusted R$^{2}$ & 0.912 &  \\ 
Log Likelihood &  & $-$57.076 \\ 
Akaike Inf. Crit. &  & 136.153 \\ 
Residual Std. Error & 0.041 (df = 102) &  \\ 
F Statistic & 117.561$^{***}$ (df = 10; 102) &  \\ 
\hline 
\hline \\[-1.8ex] 
\textit{Note:}  & \multicolumn{2}{l}{$^{*}$p$<$0.1; $^{**}$p$<$0.05; $^{***}$p$<$0.01} \\ 
 & \multicolumn{2}{l}{Data-transformed linear and logistic regression results} \\ 
\end{tabular} 
\end{table}

\begin{Shaded}
\begin{Highlighting}[]
\NormalTok{mod5<-}\StringTok{ }\KeywordTok{lm}\NormalTok{(}\DataTypeTok{data =}\NormalTok{ data_tf, yes_pct }\OperatorTok{~}\StringTok{ }\NormalTok{medage }\OperatorTok{+}\StringTok{ }\NormalTok{log_medincome }\OperatorTok{+}\StringTok{ }\NormalTok{white_pct }\OperatorTok{+}\StringTok{ }\NormalTok{trump_pct}
           \OperatorTok{+}\StringTok{ }\NormalTok{log_public_pct }\OperatorTok{+}\StringTok{ }\NormalTok{time_pct  }\OperatorTok{+}\StringTok{ }\NormalTok{sqrt_voter_turnout }\OperatorTok{+}
\StringTok{              }\NormalTok{current_plan)}
\KeywordTok{summary}\NormalTok{(mod5)}
\end{Highlighting}
\end{Shaded}

Call: lm(formula = yes\_pct \textasciitilde{} medage + log\_medincome +
white\_pct + trump\_pct + log\_public\_pct + time\_pct +
sqrt\_voter\_turnout + current\_plan, data = data\_tf)

Residuals: Min 1Q Median 3Q Max -0.099920 -0.031591 -0.003479 0.027983
0.098568

Coefficients: Estimate Std. Error t value
Pr(\textgreater\textbar t\textbar)\\
(Intercept) 1.031840 0.218778 4.716 7.64e-06 \textbf{\emph{ medage
0.003603 0.001511 2.384 0.01899 }\\
log\_medincome -0.008331 0.022302 -0.374 0.70952\\
white\_pct 0.116703 0.050746 2.300 0.02350 *\\
trump\_pct -0.856963 0.058590 -14.626 \textless{} 2e-16 }*
log\_public\_pct 0.022304 0.008263 2.699 0.00814 \textbf{ time\_pct
-0.281374 0.089698 -3.137 0.00223 } sqrt\_voter\_turnout -0.319107
0.103460 -3.084 0.00263 \textbf{ current\_plan2 0.021184 0.043430 0.488
0.62677\\
current\_plan3 -0.033367 0.011777 -2.833 0.00555 } current\_plan4
0.013583 0.011459 1.185 0.23864\\
--- Signif. codes: 0 `\emph{\textbf{' 0.001 '}' 0.01 '}' 0.05 `.' 0.1 '
' 1

Residual standard error: 0.04125 on 102 degrees of freedom Multiple
R-squared: 0.9202, Adjusted R-squared: 0.9123 F-statistic: 117.6 on 10
and 102 DF, p-value: \textless{} 2.2e-16

\begin{Shaded}
\begin{Highlighting}[]
\KeywordTok{anova}\NormalTok{(mod5,mod3)}
\end{Highlighting}
\end{Shaded}

Analysis of Variance Table

Model 1: yes\_pct \textasciitilde{} medage + log\_medincome + white\_pct
+ trump\_pct + log\_public\_pct + time\_pct + sqrt\_voter\_turnout +
current\_plan Model 2: yes\_pct \textasciitilde{} medage +
log\_medincome + white\_pct + log\_public\_pct + time\_pct + trump\_pct
+ sqrt\_voter\_turnout + current\_plan Res.Df RSS Df Sum of Sq F
Pr(\textgreater F) 1 102 0.1736\\
2 102 0.1736 0 -8.3267e-17

\begin{Shaded}
\begin{Highlighting}[]
\KeywordTok{ggplot}\NormalTok{(final_data, }\KeywordTok{aes}\NormalTok{(}\DataTypeTok{x =}\NormalTok{ trump_pct, }\DataTypeTok{y =}\NormalTok{ medincome)) }\OperatorTok{+}
\StringTok{   }\KeywordTok{geom_point}\NormalTok{(}\KeywordTok{aes}\NormalTok{(}\DataTypeTok{colour =}\NormalTok{ white_pct))}
\end{Highlighting}
\end{Shaded}

\includegraphics{Zhong_paper_files/figure-latex/unnamed-chunk-12-1.pdf}

\begin{Shaded}
\begin{Highlighting}[]
\KeywordTok{ggplot}\NormalTok{(final_data, }\KeywordTok{aes}\NormalTok{(}\DataTypeTok{x =}\NormalTok{ white_pct, }\DataTypeTok{y =}\NormalTok{ yes_pct))}\OperatorTok{+}
\StringTok{   }\KeywordTok{geom_point}\NormalTok{()}
\end{Highlighting}
\end{Shaded}

\includegraphics{Zhong_paper_files/figure-latex/unnamed-chunk-13-1.pdf}

\hypertarget{references}{%
\section{References}\label{references}}

\end{document}
